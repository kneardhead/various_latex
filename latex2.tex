
\documentclass{article}
\usepackage{amsmath}
\begin{document}
\textsf{I'am Ahmed Saad Sabit, an idiot from Chittagong, ever heard of it? If not, then the solution to this problem is left as an exercise.\\
The equation that you see is the Maxwell's Equation that is the first one. It's called the Gauss's law for Electricity}

\begin{equation}
\oint_{0}^{\pi} \sin x \, dx = 2 \notag
\\  
								\qquad	\textsf{(When conditions are met)}								
\end{equation}

It's very dark and getting \emph{deeper} with time, I do have much hardship going on with the method of typing effectively.


\newpage



\begin{equation}
P(x) = 18 x^3 + 15x^2 - x -2 
\end{equation}


\begin{equation}
\frac{\oint_{- \infty}^{\infty} \int_{ - \frac{\pi}{2}} ^{ \frac{\pi}{2} }
\frac{\sin x \times \sqrt{x^2 y^2 z^3}}{x^3 + y^3 + z^3 + 8xyz^{\gamma}}
\,
\partial x  \, \partial y}{\sum_{x}^{\infty} \vec{R} }
\end{equation}


We can nicely do another kind of writing over the in style,
\begin{equation}
\int_a^b	u \frac{d^2v}{dx^2} \, dx
=	\left. u \frac{dv}{dx} \right|_a^b 
-
\int_a^b \frac{du}{dx} \, \frac{dv}{dx} \, dx
\end{equation}
The thing that fascinates me the most is that the math can be edited in almost no time in the latex system, yes of course, given you have an insane speed in hands and type relatively faster and know how to use the pinky finger as frequently as possible, the \emph{unit vector} that makes the system go possible is the $\vec T$ that emerge from the fingers, each assigned by a sub $\vec T , \,\vec T_1 , \, \vec T_2, \, \vec T_3, \, ...$ vectors.
Now there is the method how a command is imposed on the tex editor. For doing a vector, what I need to type is
\verb \vec{T}|
and we are done





\end{document}