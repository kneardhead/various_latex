\documentclass[10pt]{article}
\usepackage{amsmath,graphicx}
\usepackage[top=0.5in, bottom=0.5in, left=0.95in, right=0.95in]{geometry}
\title{The Solution to the Stupid Mechanics Problem}
\author{Ahmed Saad Sabit}
\date{\today}
\clearpage

\begin{document}
\maketitle

\textbf{\textsf{Problem :}} \, A wedge with mass M and wedge angle $\theta$ sits on a frictionless surface with a block of mass m. If the block starts to slip, then find the acceleration of the wedge.\\
\\
\begin{figure}[hbtp]
\centering
\includegraphics[scale=0.49]{M1 Fig1.png}
\caption{The diagram free body}
\end{figure}
\textbf{\textsf{Idea's and Solution:}} \, The main idea is that the block has a normal force that pushes the wedge by an incline. The y-component of that reaction force doesn't but the x parallel one comes to accelerate the wedge. The advanced additional insight is to understand that the \emph{Normal force} governs the motion of the system, but we cannot write by our intuition that $N=mg\,cos\theta$ because we have an \emph{accelerating} block. Hence take N as an unknown as we write our equations. \\
\\
We should make our problem simpler to use in our head by taking all components by the ground, not along the inclined surface (specially for the block). \textbf{Target variable is A}.
\\ 
\\
\\
Writing the good ol' F= ma for the things.You'll notice the last equation is little teensy if you're novice like me.
\begin{align}
mg - N\,cos\theta \,&= ma_y \\
N\,sin\theta &= ma_x \\
N\, sin\theta &= MA \\
\frac{a_y}{ a_x + A} &= tan\theta 
\end{align}
\\
\begin{figure}[hbtp]
\centering
\includegraphics[scale=0.47]{M2 Fig2.png}
\caption{Note that there is a horrible looking $\theta$}
\end{figure}
\\
Till the equation 3 above we are fine. But we notice we have 4 unknowns $N,\,a_x,\,a_y,\,A$ but 3 equations. So, remembering \textbf{As many unknown that many equations} we need another independent equation. This can be found by knowing the constraints, as Prof. Morin taught.\\
\\
The main idea of finding the last equation is \emph{changing the reference frame} into the block's one. Because the block is (in our general frame) accelerating towards the right in $a_x$, in it's frame we'll find that the whole universe accelerates towards the left with the same $a_x$ (common physics fact). At the same time the block is accelerating in the left with A respect to the ground, so we sitting on the block find whole ground move at $a_x$ and the block moving relative to the ground at A. So, relative to the block, the wedge accelerate at $(a_x + A)$. \\
\\
Keeping this in mind, we look for the constraint. If the block comes down by a y (as in the figure), then the wedge to keep up with the block has to move x leftwards. The relation of x and y is clear, it has to make up with the incline as in the figure, and looking at that triangle aside, we can notice that 
\[x \, tan\theta = y\] \\
And differentiating twice,
\[ \frac{d^2 x}{dt^2} tan\theta = \frac{d^2 y}{dt^2}\] \\
Working in the block's frame, the $\frac{d^2 x}{dt^2}$ is acceleration of the wedge along x direction. And $\frac{d^2 x}{dt^2}$ is acceleration of the block along the y direction, refer to the figure. This evidently leads us to write that
\begin{equation}
(a_x + A) \, tan\theta = a_y
\quad \longrightarrow \quad
\frac{a_y}{a_x + A} = tan\theta 
\end{equation}
\\
So we have made 4 okay equations. 
\\
\\
\textbf{\textsf{A mistake I made before :}} (also being a motivation to waste time behind this document), I the first time chased the problem making a promise not to cheat by using \emph{Analytical Mechanics}. The last equation what started making me crossed off, I idiotically thought that $a_x$ points the right, A points the left, why won't I subtract to get the relative acceleration ? This was the outcome of \textbf{not being flexible to efficiently chase and switch the frame of reference}. My wrong thought can neither be applicable to a wedge frame. Plus initially I couldn't convince myself that an equation from a different frame might really work, but Physics at least, wasn't so cruel. Rather it was more appreciable that the 4th equation is barely a ratio that outline a simple $tan\theta$. So, using that equation was legal. Better you not make the mistake as me!
\\
\\
So you have these equation's and I leave it as a worthwhile algebra problem.\\ 
\\
\textbf{\textsf{Answer:}} $A = \frac{mg\, sin\theta \, cos \theta}{M + m\, sin^2\theta}$
\\
\\
\textbf{\textsf{Reference:}} \textsf{ An Introduction to Classical Mechanics, David Morin, \textbf{Problem 2.7 (2. Moving Plane ***)}}















\end{document}











