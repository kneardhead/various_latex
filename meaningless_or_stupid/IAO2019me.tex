\documentclass[11pt,a4paper]{article}
\usepackage[utf8]{inputenc}
\usepackage{amsmath}
\usepackage{amsfonts}
\usepackage{amssymb}
\usepackage[left=2cm,right=2cm,top=2.8cm,bottom=2cm]{geometry}
\author{Ahmed Saad Sabit}
\title{XXIV International Astronomy Olympiad 2019}
\begin{document}
\maketitle
The 24th edition of the International Astronomy Olympiad was going to be held in Piatra Neamt, Romania from 17th to 27th October, 2019. Our delegation was selected after a simple selection test. The Alpha category students in camp were 3, where Beta were 18 or so. Thus the Alpha (with me) were a bit too lucky.\footnote{Thanks to Almighty for sure!}
\section{The Team}
We were a team of 8 members from Bangladesh.  \\
The students
\begin{enumerate}
\item Sajia Shahrin Neha - BD $\beta$ -1
\item Pallob Kanti Paul - BD $\beta$ -2
\item Anam Bin Morshed - BD $\alpha$ -1
\item Ahmed Saad Sabit - BD $\alpha$ -2
\item Md. Atif Absar - BD $\alpha$ -3
\end{enumerate}
The elders
\begin{enumerate}
\item Moshurl Amin - Team Leader
\item Rubayat Hossain - Team Leader and Jury
\item Soumitra Kumar Biswas - Observer
\end{enumerate}
\section{Departure}
\subsection{To Dhaka}
I start off for Dhaka with the luggage and Dad beside me. A lot of things are swirling in my head, that's a huge load, big responsibility blah blah. But one thing that even it was my first journey out the country, that didn't make me nervous or excited. I honestly felt nothing thinking that I was supposed to go Europe. That's perspective though. Plus I thought that (wrongly) IAO won't require superhard calculations. I needed head chilled. 

We get into the lower one of the double decker sleeper coach with Fundamental Astronomy on my hand, it's a nice feeling to keep a book in hand inside a crowd. Cause this case we don't have to think if we really fit into the place plus I wanted even the last bit of theoretical osmosis of the eclipse phenomenon.\footnote{Plus it happened that it was my doom in Theoretical Exam} The seat was cozy, and I got to learn some points. The Stellarium was a help too, I remember how the eclipse looks from various latitudes of the Earth. But sad, refer again to footnote 1. I stopped Astronomy for a while and got into the messenger working out a problem of a Estonian Finnish Kinematics problem of the skiers possible most least time to change the trajectory by $90\deg$. I couldn't do much forward on the problem as usual but keep the problem on the group. Discussion off because it was 2 or 3 of the night.

Reaching we get to Aunt ``Khoku"'s house to relax there a moment before getting into the pre-departure camp. \footnote{Call it a meetup, we weren't taught anything much.} And I stay up with the astronomy texts.
\subsection{To Istanbul}
We wear the official yellow colored simple T-Shirt and all get into the vehicle to Airport. Boro Mama (Elder uncle) was intimidated and concerned of the case I lose the wallet from my past doings. After waiting a six or seven hours we got on the plane. It wasn't boring as we were together, also Fatih, a Turkish explained his engineering job in a beverage factory, the  AKIJ near Dhaka. And yes, he bitterly complained the awkward and lame nature of the Bangladesh people he had to deal with. But what I recall the most is his explanation of Turkish Scientists, Cahit Arf; and pronouncing Einstein as ``Ein-shtain'' from the Turkish accent. On the plane it was okay. The taking off was little scary at first but I didn't think about it anymore as under a minute I got used to it. At least,  my shelter is inside the Air Force base; a few hundered meters from the Air strip. Pallob slept dead all the 6000 kilometers.
\subsection{Istanbul Airport}
What? It was \begin{huge}
\textbf{HUGE}
\end{huge}. Too big and long. The `Mescid' was nice there. Me and Pallob did astronomy problems as the others were out looking forward to spend the Euros. The food there was little problematic, but I know in Istanbul it was better outside rather than the airport. Good thing is there were couch to sit. And it was a good wait.
\subsection{To Bucharest}
We reach the plane and Pallob waits to sleep but this time I don't let that happpen. But Neha did and Pallob did some work clicking her photos then. Relativly it was a shorter journey on air. Later I came to know my family with the help of Boro Mama tracked the Plane all the time, weird. 

We land on Bucharest and feel the cold. I felt little uneasy but soon started to like the cool air into the lungs. It was pure with assumed $3-5 \, ppm$ air. So my high immunity problems didn't bother anymore.\footnote{Quick Biology: High Immunity (more than necessary) is Allergy, which I have with dust.} Then go to the train station by bus and get into McDonald's after a few minutes where we meet a boy who was waiting for someone to come. Inside the McDonald's I got into problem with what meat was it. We met the Nepalese team and their awesome team leader Suresh Sir. He was a guy of awesome. There we also met a Physics person working on Renewable things who termed it His Life. That was something nerd after a tens of hours. 

We struggled a little with the train but Romanians were so simple and straight forward opposed to what we are accustomed of the Bangladeshis that it didn't matter much. We were okay. This moment that boy we met waiting was going to another compartment passing us where we bade him Hi and Neha asks if she came, the boy had the smile on him gone and pushed himself through the door suddenly, without speaking a word which was a sign that she wasn't really there. It made me fell little sad for him but Neha I guess didn't see it by the seats. An old man tried heart and soul communicating us, fortunately Rubayet Vaiya (I mean in English Rubayet Bro) could translate his signs to us. The man was really emotional and heart broken as he didn't know English to communicate us.  

The train landed in Succeva from where we were supposed to go to Piatra Neamt by some other train. Belive me, that other train was totally Harry Potter like where Harry and Ron enjoy Chocolates. We were stuffed with bags and luggage and uttering things in Bengali. The Romanians near couldn't get us. It was a thing, we could say whatever but they don't get us. We finally came to the breezy and cold Piatra Neamt by around half an hour of the Local Fazr there.

We were welcomed by a Blue Haired Woman who probably was the local organizer or some sort of that. We also were warmly welcomed by the hotel itself, the goodies and the identification card was given to us there. And we got into the room and I had (we had) a home type room, I don't know what is that called, for three person, one room with a single bed and another with two. I got the single, as I was well determined, I won't be getting in a mess so fast. And then starts the real deal.

Oh, I also shortly text Vlad Rosca who seem to like the Blue cap given against the notebooks and some travel guides of Piatra Neamt. To tell, Vlad is last year IAO bronzer from Romania and quite interestingly this year APHO. What a dude!

\emph{(I am squinting and struggling with the keyboard})

\section{The game}
Rather than these fuzzy unimportant journey talks I'd rather like to have an analysis of the Problems. So we had the triples
\begin{itemize}
\item The Theoretical Round
\item The Observational Round
\item The Practical Round
\end{itemize}
Me and Pallob were obsessed with the fact of doing well in the Theoretical Round, so we were doing some Keplar Orbit problems in Istanbul. As we were given the problems in the examination room, what I thought was ``GOTCHA", as it came that the first problem was of the moon. It was somewhat repelling for me as I didn't got to find any useful resource to understand it's mechanics quite nicely. I started to have some little ideas on the problem, making a statement \emph{As Pitra Neamt has a quite high latitude, so what we should expect that the Sun and Moon declination and other coordinates (equatorial) will have a discrepancy.} So I started to do the maths of finding the difference, but ALAS! I got lost into the maths that made it clear that I can't make a nice standout anymore, though I had some hours left!

I at the end didn't came to any conclusion but aimlessly concluded my maths unsolved. Later speaking to Dananjay Raman, India and later absolute winner in the $\alpha$, told me started me out with the assumption \emph{Moon and Sun declination are same...} It was disappointing. I guess that amount of picky you become if you ever looked 200 problems in Physics book, you have to be picky.  

As I was having nightmares during the exam, what I thought a moment that ``It's now or it's never", and It's me a lucky duffer who hadn't been helped with these sort of motivation. Moving to the NGC problem made me think of a dimensional mess. Because if the distribution of the stars are uniform and the observer is in front someplace nearby that of course, the star directly in front of him cannot let him see the center star, and the distribution is \emph{Volumetric}, hence I didn't know the way to do something. It was getting worse, the brightness also make the vision too opaque for an observer to see the stars nicely. I did some maths I've forgot by now, but they of course didn't do much progress as they needed to. I was arrogant of thinking myself taking measurements of the image and making scales, I guess my scaling was wrong too. The bad luck was erupting to a greater value.
\\
There was a comet problem, but being so emotionally fragile couldn't make out that simple Physics. I could have done that other times under a minute, but the devastation of the first two problem did not let that. \\

The simultaneous sun rise problem in Chukotka was somewhat creepy to me at first, as the value of the difference of the longitude comes around 2 degrees, simply the value had problem. I tried to do with the problem statement of the exactly opposite side on the Earth. I was feeling awkward with the way the figure came out as I gave up and started to draw Grizz and Ice Bear on two sides of the Earth. It felt to me, ``Oh dang this phenomenon is very unique and the sun diameter (angular measure) is making a drastic position in the Physics and the Geometry of the system." AND LATER as we were in the Chinese team room, asking to see some of the way they write maths, Hu gave his theory solution copy where the Chukotka problem solution started saying ``SITUATION IMPOSSIBLE." I wonder how sad my life could be!\\
There was almost no progress made by me as I went through, I felt so bad after we got out the theory exam!













\end{document}