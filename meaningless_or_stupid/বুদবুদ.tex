\documentclass[11pt,a4paper]{article}
\usepackage{amsmath}
\usepackage{amsfonts}
\usepackage{amssymb}
\usepackage{graphicx}
\usepackage{tcolorbox}
\tcbuselibrary{breakable}
\usepackage[banglamainfont=Kalpurush,banglattfont=Siyam Rupali]{latexbangla}
\begin{document}
\textbf{একটা বুদবুদ এবং তার নাড়াচাড়া । } \\ 
{\tiny Part 1 of 3.} \\
ভর, দড়ি, স্প্রিং নিয়ে আমরা প্রচুর অঙ্ক কষলাম, এবার একবার ভেবে দেখি একটি বুদবুদের পেছনের পদার্থবিজ্ঞান কেমন হতে পারে? 

বুদবুদ অথবা বাবল নিয়ে আমরা এমনে কম বেশি জানি, বাতাসের একটা ছোট্ট চেম্বার (বা ঘর) যেটা পানির ভেতর গোলাকার হয়ে থাকে। পেপসি কোকা-কোলার বোতলে এটা ভালই লক্ষ্য করা যায়। 
বুদবুদগুলোর ভেতরে বাতাস ভরা থাকে, পানি এ বুদবুদের চারপাশ দিয়ে চায় চাপ দিতে।

চাপ মানে মনে আছে? প্রতি একক ক্ষেত্রে যে পরিমাণ বল প্রয়োগ করা হয় তাই চাপ। অথবা এভাবে বলি, প্রতি এক বর্গ মিটার ক্ষেত্রে ১ নিউটন বল প্রয়োগ করা হলে চাপ হবে ১ প্যাসকেল। 
তাহলে, এই চাপ বুদবুদ সহ্য করে কেমনে? চাপের চোটে বুদবুদের ভেতরকার বাতাস একটু সঙ্কুচিত হয়। এবং সঙ্কুচিত হলে বাতাস আবার বাইরের দিকে চাপ দেয়া শুরু করে তখন এমন অবস্থা তৈরি হয় যে পানির দেয়া চাপ এবন গ্যাসের বাইরের দিকে প্রয়োগ করা চাপ সমান হয়ে যায়। চাপ যেহেতু সমান তাই এখানে নাড়াচাড়ার আর কিছু নেই, কারণ একটা চাপ আরেকটাকে বিপরীতভাবে ক্রিয়া করছে, তাই বুদবুদটি যেমন আছে তেমনি থাকবার চেষ্টা করে। বুদবুদের মধ্যকার গ্যসের তৈরি করা চাপ অনেকটা স্প্রিঙকে সঙ্কুচিত করলে সেটার বাইরের দিকে বল দেয়ার মতন, তুমি দুটি কে একেওপরের মতন ধরে নিতেই পারো! 

\begin{figure}[hbtp]
\centering
\includegraphics[width  = 0.4 \textwidth]{../HCampwork/paradox/OIP.jpg}
\caption{Random Bubbles }
\end{figure}
এখন আমরা এই বুদবুদের এখানে ওখানে যাতায়াত করা নিয়েই একটু মনোযোগ দিব। 
আর্কিমিডিসের তত্বের কথা মনে করি, একটা বস্তুকে যখন কোনো তরলে (অথবা গ্যাস) রেখে দেয়া হয়, তখন বস্তুটি কিছু পরিমাণ তরল সরিয়ে দেয়। বস্তুর যে অংশ পানির মধ্যে ডুবে থাকে তত পরিমাণ তরল সরে যায়। ঠিক যতটুকু পরিমাণ তরল সেটা সরায়, ঠিক ততটুকু "তরলের ওজন সমানের" বল বস্তুকে উপরে ঠেলে দিতে চায়। এই বলই হলো বোয়ান্ট ফোর্স, অথবা প্লবতা। কারণ বস্তু যে পরিমাণ তরল সরিয়ে দেয়, সে তরল আবার নিজের আগের জায়গায় ফিরে আসতে চায়, তা করতে হলে বস্তুকে আবার উপরে ঠেলে দেয়া ছাড়া উপায় নেই।  
তাই বস্তু পানিতে দুটো বলের মধ্যে থাকে, একটা তার নিজের ওজন, এবং আরেকটা প্লবতার। ওজন তাকে নিচের দিকে টানে, প্লবটা উপরের দিকে ঠেলে। নৌকার ক্ষেত্রে এই প্লবতার মান এবং ওজন সমান হয় যায় বলে একটা নৌকা ডুবে না। নৌকাটাকে কেউ যদি জোর করে আরেকটু ডুবিয়ে দেয়, তাহলে প্লবতা বেড়ে যাবে, কারণ অতিরিক্ত পানি সরে যাচ্ছে, এখন ওজনের মান তো ধ্রুবক, তাই প্লবতা ওজনের বেশি হলে নৌকা আবার উপরের দিকে উঠে যেতে চায়। কারণ প্লবতার বল কাজ করে উপররে দিকে। 
\begin{figure}[hbtp]
\centering
\includegraphics[width = 0.5\textwidth]{../HCampwork/paradox/1200pxBuoyancy.png}
\caption{প্লবতা }
\end{figure}

কি হবে যদি নৌকার ভর অনেক কম হয়? তাহলে ওজন কম হবে। সুতরাং অল্প কিছু পানি সরিয়ে দিয়েই নৌকা ভালো করে ভেসে থাকবে, নৌকার তল খুব একটা ডুবে যাবে না। 
তাহলে একটা বুদবুদের ক্ষেত্রে কি এই নিয়ম খাটবে? হ্যাঁ, অবশ্যই। কিন্তু তখনই যখন বুদবুদ পুরোপুরি পানির নিচে থাকবে। পানির উপরে এসে গেলে তো আর বুদবুদ থাকছেনা! সেটার মধ্যকার গ্যাস চারপাশের বাতাসের সাথে মিলিয়ে যাবে তখন। 
বুদবুদ পানির ভেতর থাকলে সেটা বেশ ভালই প্লবতার সম্মুখীন হবে, কারণ একটা বুদবুদের ওজন নিতান্তই কম, সত্যিকারে অনেক কম। 
ঘনত্বের কথায় এসে যায়, একটা বস্তুর ভর দিয়ে তার আয়তন ভাগ দিলেই ঘনত্ব আসে। এভাবেও ভাবা যায়, "এই বস্তুর প্রতি এক বর্গ মিটার আয়তনের ভর কত?" 
একটা বুদবুদের মধ্যকার বাতাসের ঘনত্ব যদি হয় $n$, আর আয়তন যদি হয় $V$, তাহলে তার ভর হবে$ m = nV$ । এবং ওজন? সহজ, $W = mg = (nV)g = nVg$।
এবং কতটুকু পানি সে সরাবে? সরিয়ে দেয়া পানির আয়তন বুদবুদের আয়তনের সমান হবে, তাই সে পানির ঘনত্ব যদি N হয়, তাহলে সে পানির ওজন হবে 
$W = mg = (NV)g = NVg$ 
বলেছিনা প্লবতার বল তাক করবে উপরের দিকে? (যেহেতু বস্তুকে তরল সরিয়ে নিজের পুর্ববস্থানে আসঃতে চায়) এবং ওজন তাক করে নিজের দিকে (পৃথিবীর ভূমির দিকে)। সুতরাং লব্ধি বল বা Total Force হবে, 
\[F = NVg – nVg\] 
এবং বুদবুদের ওজন তার সরিয়ে দেয়া পানির চাইতে কম বলে এই লব্ধি বল কাজ করবে উপরের দিকে, মানে তাক করবে উপরের দিকে। এবং এটা আসলে ঠিক, একটা পেপসির বোতলের বুদবুদগুলো উপরের দিকেই উঠতে চায়। 
আচ্ছা, এই বুদবুদ উপরে উঠতে গেলে এর ত্বরণ কতো হতে পারে? একটা আন্দাজ করি এখানে, 
নিউটনের দ্বিতীয় সূত্র ব্যবহার করে আমরা সাধারণত বের করি এতো এতো বল প্রয়োগ করা হলে এতো ভরের বস্তু কত বেশি ত্বরণ নিয়ে এগিয়ে যাবে, সহজে দেখাতে গেলে, 
\[F = ma\] 
এখন আমরা তো ঘনত্ব নিয়েই অঙ্ক চালাচ্ছিলাম উপরে, তাই না? আমরা পেয়েছিলাম যে বুদবুদের ভর হবে m=nV, তাই আমরা পানির নিচে বুদবুদের লব্ধি বলের মধ্যে নিউটনের দ্বিতীয় সূত্র বসিয়ে দিলে পাই, 
\[ma= NVg – nVg\]
এর থেকে আসে, 
\[(nV)a = NVg – nVg \]
\[nVa = NVg – nVg \] 
দুপাশে V কেটে দিলে পাই, 
\[na = Ng – ng\]
এখন একটা কথা আছে, আমরা দুদিকে সরাসরি n ভাগ দিয়ে a বের করে ফেলতে পারি। কিন্তু একটু ভেবে দেখি, বাতাসের তুলনায় পানির ঘনত্ব কত? 
সাধারণ অবস্থায় বাতাসের ঘনত্ব ১.২০ কেজি প্রতি ঘন মিটার (একদম সরাসরি বলতে গেলে ২০ ডিগ্রি সেলসিয়াস তাপমাত্রায় সাধারণ পরিবেশে প্রায় ১০০০০ প্যাসকেল সাধারণ বায়ুচাপে)। আর পানির ক্ষেত্রে তা ১০০০ কেজি প্রতি ঘন মিটার। ১০০০ গুন বেশি। বেশ বড় পার্থক্য। তাই, বাতাসের ঘনত্ব বলতে গেলে পানির কাছে কিছুই না, তাই $Ng – ng$ তে এই n এর মান (বাতাসের ঘনত্ব) অনেক কম। তাই এটার Ng থেকে যোগ বিয়োগ খুব বেশি যায় আসেনা। তাই আমাদের ng এখানে লেখবারই কোন দরকার নেই, কারণ একটা ছোট সংখ্যা কে আমরা একটা বেশ বড় সংখ্যার সাথে বিয়োগ করছি। অনেকটা $১০২৩.০২৫ – ০.০০১ = ১০২৩.০২৪$ এর মত ব্যাপার, এখানে এই ০.০০১ এর মূল্যই নেই। তাই লেখারও দরকার নাই, কারণ ১০২৩.০২৫ এবং ১০২৩.০২৪ প্রায় সমান। সুতরাং,  
\[na = Ng\] 
এখন আমরা পাই, 
\[a= N/n g\] 
এখন পানির ঘনত্ব$ N=1000 kg/m^3$ আর$ n = 1.2 kg/m^3$ । খেয়াল করো $N/n = 1000$ প্রায়। তাহলে উত্তর কি দাঁড়ায়?
\[a= 1000g = 1000 (9.8) \, m/s^2 = 9800\,  m/s^2 \]
এটার উত্তরটা আসলেই অদ্ভুত, আমরা জানি একটা বস্তু খোলা আকাশে মাটিতে ৯.৮ মিটার প্রতি বর্গ সেকেন্ডে নিচে পড়ে, আর এখানে একটা বুদবুদ তার চাইতে ১০০০ গুন বেশি ধ্রুত পানিতে উপরের দিকে ত্বরন্বিত হচ্ছে! মানে একদম একপলকের মধ্যে একটা বুদবুদের বোতলের গা বেয়ে ধারাম করে পানিতে উঠে আসার কথা! অথচ আমরা দেখি যে একটা পেপসির বোতলে এতো ধ্রুত কখনো বুদবুদ উঠেই না! 
কীরে, তাহলে ভুলটা হলো কই? 
পরের পার্টে এই ঝামেলার সমাধান করবো, কিন্তু তুমি ততক্ষণ ভেবে কি বলতে পারবে, কেন উত্তর এমন উদ্ভট আসলো? অবশ্যই কমেন্টে জানাবে। এবং এখানে ম্যথ করতে করতে আমরা কিন্তু একটা গুরুত্বপুর্ন কথা হিসাবই করিনি (অবশ্য তাও কিন্তু আমাদের এ পর্যন্ত অঙ্কে অদ্ভট উত্তর ছাড়া ভুল নেই) । বলতে পারবে সেটা কী? 
\newpage
\textbf{একটা বুদবুদ এবং তার নাড়াচাড়া । } \\ 
{\tiny Part 2 of 3.} \\
প্রথমে আগের পুরনো উত্তরগুলি দিই। তাহলে আমাদের আগে সমস্যা হয়েছিল কোথায়? সমস্যা ছিল যে উত্তর স্বাভাবিকের চাইতে বেশি আসছিল। 

কারণ হলো মূলত দুটি,
\begin{itemize}
 \item যদি বুদবুদ উঠতেই চায় তাহলে তাকে উপরের পানি বা তরলকে সরিয়ে উঠে যেতে হবে। যদি তরলকে সরানো হয় তাহলে তরল নিজের সাথে ঘর্ষনের সম্মুখীন হয়, যাকে আমরা বলি \textbf{Viscosity অথবা সান্দ্রতা} ।  
 \item বুদবুদ কিন্তু তরলের নিচেও নিজের আকার ধরে রাখে, তাই যখন বুদবুদ উপরে উঠতে থাকে, তখন বুদবুদ কে তার \textbf{সামনে কিছু পরিমাণ তরলকে নিয়েও এগিয়ে} যেতে হয়। এই তরল আগানোর সাথে সাথে বুদবুদের গা বেয়ে সরে যায়, কিন্তু আরো তরল এসে বুদবুদের সামনে জমা হতে থাকে। তাই বুদবুদের উপরের সাথে লেগে থাকা তরলের ভর মোটামুটি সমান থাকে। যেহেতু বুদবুদ কে সামান্য বেশি তরল নিয়ে এগিয়ে যেতে হয়, তখন বুদবুদের ত্বরণ স্বাভাবিকভাবেই কমে যায়।
\end{itemize}
বলে রাখা উচিত বেশ কিছু ক্ষেত্রে আরো দু তিনটা বেশি কারণ থাকতেই পারে যা তরলের উপর নির্ভর করে। এখন এমনও হয় যে অতিরিক্ত পানিকে ঠেলে নিতে যে বল প্রয়োজন, তার চাইতে সান্দ্রতা বা ভিস্কসিটির বলের পরিমাণ কম। এমনও হয় যে অনেক ক্ষেত্রে সান্দ্রতা কে হিসাবি করা হয় না কারণ তার মান কম থাকে। সুতরাং এই অতিরিক্ত তরলের উপর ই আপাতত মনোযোগ দিই। 

তাহলে আমরা এবার বুদবুদ-তরল কে হিসাব না করে একটা গোলক বল এবং পানির কথাই আমলে আনি। 

ধরি আমরা এই বলকে পানিতে ডুবিয়ে সোজা যেকোনো দিকে এগিয়ে নিয়ে যাচ্ছি। তাহলে এই বলের অতিরিক্ত পরিমাণ পানিও ঠেলে নেয়া উচিত, এই অতিরিক্ত পানির ভর বলের ভরের সাথে যোগ হবে। যোগফলকে আমরা বলবো Effective Mass এবং অতিরিক্ত পানির ভর কে বলবো Added Mass। 
\begin{figure}[hbtp]
\centering
\includegraphics[width = 0.8\textwidth]{../UaDrawings/PNGCairo/addedMaSss.png}
\caption{Diagram showing added Mass}
\end{figure}
\textit{আমরা কী কোনোভাবে এই এডেড মাস বের করতে পারি?} 

হুবহু করে এই সমাধান বের করতে হলে আমাদের জানতে হবে ঠিক কতটুকু পানি এই বলের সাথে প্রায় সমান গতিতে এগিয়ে যাচ্ছে, শুরু করতে হলে আমরা লিখব অয়লারের সুতরও,
\[ \rho \frac{d \vec{v}}{dt } = \vec{f} - \nabla p \]
দেখেই বোঝা যাচ্ছে এই ভাবে আজকে অঙ্ক করলে শেষ হবে না, আমাদের কে বাউন্ডারি কন্ডিশন সেট করে অঙ্ক চালাতে হবে, তবে এত জটিল হিসাবের কোন দরকার নেই। 

আমরা একটু ডাইমেনশনালি চিন্তা করি। 
\begin{itemize}
\item Added Mass বা অতিরিক্ত পানি নির্ভর করবে বলের \textbf{আকৃতির} উপর । 
\item Added Mass বা অতিরিক্ত পানি নির্ভর করবে বলের \textbf{আয়তনের} উপর ।
\item Added Mass বা অতিরিক্ত পানি নির্ভর করতে পারে বলের ও পানির নিজের আরো কিছু বিশিষ্টের উপর। 
\end{itemize}
এখন সুবিধার জন্যে আমরা শেষের পয়েন্ট নিয়ে বেশি মাথা ঘামাবো না। আমি এই সুত্রে একটি মেকানিক্সের প্রব্লমে এখানে বসিয়ে দিই। 
\begin{tcolorbox}[breakable]
\textbf{Problem :} A metallic sphere of radius $r$ and density $\rho$ is moving in water, it is freely falling in an acceleration $a_0$. The water density is $\rho _w$. Assume the water flow around the ball is not chaotic (aka.\footnote{Also Known As*} Laminar). 

\textit{What will be the acceleration of a spherical bubble} of radius $r_{bub}$ ? \\
\textbf{সমস্যাঃ} একটি লোহার গোলক যার ব্যসার্ধ $r$ আর $\rho$ ঘনত্ব; পানিতে ডুবে যাচ্ছে $a_0$ ত্বরণে। পানির ঘনত্ব $\rho _w$ । ধরে নাও পানির প্রবাহ সরল এবং স্বাভাবিক। \\
\textit{তাহলে পানির নিচে একটি বুদবুদের ত্বরণ কত হবে? }
\end{tcolorbox}
আসলে এর উত্তর আমরা আগেও বের করেছি, তখন উত্তর উলটাপালটা আসছিলো, এখন অবশ্য বেশ সচেতন হয় গেছি। আসো বের করি এটার সমাধান আমরা কেমনে বের করব। \\

\textbf{সমাধানঃ} \\
প্রথমে ভেবে দেখ যে আমাদের করতে হবে কী? 
\begin{center}
পানির নিচে বুদবুদের উপরে উঠার ত্বরণ কত? 
\end{center}
\textbf{আপাতত বোঝার স্বার্থে এবং লেখার সুবিদার্থে আমি $\rho$ দিয়ে বুদবুদের বাতাসের ঘনত্ব বোঝাবো এবং শেষে গিয়ে ঘোষণা দিয়ে আবার লোহের ঘনত্ব বোঝাবো। }
এখন একটু ভেবে দেখলেই বোঝা যাবে যে একদম সরাসরি আমরা ত্বরণ বের করতে পারব না। আমাদের কে বের করতে হবে যে \textbf{বুদবুদের উপর প্রযুক্ত বলের মান কত?} তাহলে যেটা হবে, তা হলো নিউটনের সেকেন্ড ল (অনেকে N2L ও বলে) থেকে $a$ বের করে আনা যাবে। 
\[ F = ma \]
F টা কী হতে পারে?

বুদবুদের উপর কী কী কাজ করছে তা একটা ভালো ছবি একেই দেখা যায় । 
\begin{figure}[hbtp]
\centering
\includegraphics[width = 0.6\textwidth ]{../UaDrawings/PNGCairo/Bubble.png}
\caption{Forces on Bubble}
\end{figure}
তাহলে অতিরিক্ত পানির আয়তন ধরি $\tilde{V}$, তাহলে এই পানির ভর হবে $m = \rho _w \tilde{V}$। 

এখন এই বুদবুদকে যদি উপরে উঠতেই হয়, তাহলে অবশ্যই প্লবতা বেশি। আমরা ধরে নিয় যে বুদবুদের আয়তন $V$, তাহলে এটা সহজে $r$ থেকে গোলকের আয়তন এর সূত্র থেকে আমরা বের করে নিতে পারবো। এখন প্লবতার বল হবে, 
\[ F_B = V  \rho_w  g - V \rho g  \] 
এখন এই প্লবতার বল কাকে ঠেলবে?

প্লবতার বল ঠেলবে বুদবুদকে এবং অতিরিক্ত পানিকে। আগে থেকেই জানি যে বুদবুদের ওজন যথেষ্ট কম। 

প্লবতার বল হবে সরিয়ে দেয়া পানি যত বল প্রয়োগ করছে এবং বিয়োগ বুদবুদের ভর, তাহলে,  
\[ F = V\rho_w g - V \rho g \]
তাহলে, আমরা এবার বের করি কার ত্বরণ কত করে। 

অতিরিক্ত পানির ভর হল $m = \rho _w \tilde{V}$ এবং বুদবুদের  ভর হবে $m = V \rho$, তাহলে, যদি এদের ত্বরণ হয় $a$, এখান থেকে নিউটনের N2L দিলে আসে, 
\[  F=  (V \rho + \tilde{V}\rho_w) a \] 
এবং এই বলের উৎস হচ্ছে প্লবতার বল, 
\[(V \rho + \tilde{V}\rho_w) a =  V\rho_w g - V \rho g \]
এখানে আমরা শুধুমাত্র $\tilde{V}$ বাদে সবার মান জানি, তাহলে $a$ কে এক সাইডে এনে আমরা মান বসিয়ে অঙ্ক শেষ করে দিতে পারি, কিন্তু কথা হচ্ছে, আমরা $\tilde{V}$ জানিনা। 

কিন্তু খেয়াল করো প্রশ্নে একটা মেটাল (লোহার) বলের মান দেয়া আছে, সেখানে কিন্তু $\tilde{V}$ বের করার যথেষ্ট উপায় আছে। 

এবার এডেড মাস অথবা এই অতিরিক্ত পানির কথায় মন দিই। এটা যেহেতু পানির মধ্যে ডুবন্ত বস্তুর আকৃতি এবং আয়তনের উপর নির্ভরশিল, তাই আমরা বলি যে এই $\tilde{V}$ (যেটা হচ্ছে অতিরিক্ত পানির আয়তন) হবে বস্তুর আয়তনের সমানুপাতিক। অতএব, আমরা এটা লিখতে পারি যে, 
\[\tilde{V} = \alpha V \] 
অঙ্ক সহজ হলো, কেন? শেষের সমীকরণ ছিল যে, 
\[(V \rho + \tilde{V}\rho_w) a =  V\rho_w g - V \rho g \]
এই $\alpha$ কিন্তু শুধু বস্তুর আকৃতির উপর নির্ভর করে, তাই একটা গোলকের জন্যে এটা যেমন, একটা বড় গোলকের জন্যেও এটা ঠিক তেমন। একটা কিউবের জন্যে অবশ্য একটু আলাদা হবে।  যাই হোক, আমরা অঙ্ক করতে থাকি। 


এবার এটাতে আমরা, 
\[(V \rho + \alpha V \rho_w) a =  V\rho_w g - V \rho g \quad \quad 
( \rho + \alpha \rho_w) a =  \rho_w g -  \rho g \] 
যদি এটা আমরা লোহার বলের ক্ষেত্রে চালাই দিই তাহলে আমাদের $\alpha$ বের করা অনেক সুবিধার হয় যাবে। প্রথম পার্ট (প্রবলেম) থেকে আমরা লিখতে পারি। 
\[( \rho_{\text{মেটাল}} + \alpha \rho_w) a_0 =  \rho_w g -  \rho_{\text{মেটাল}}  g \] 
এখান থেকে আমরা সমীকরণের দুদিকে $a_0$ দ্বারা ভাগ করবো এবং $\rho_{\text{মেটাল}}$ দ্বারা বিয়োগ করে এক সাইডে $\alpha$ কে আনবো, তাহলে দাঁড়ায়,
\[\alpha = \frac{1}{\rho_0} \left( \frac{\rho_{\text{মেটাল}} g - \rho _0 g}{a_o} - \rho_{\text{মেটাল}} \right) 
\]
যদি সরাসরি এই এক্সপেরিমেন্ট করা হয়, তাহলে আসে যে $a_0 = 0.57 g$, যেখানে এই $g = 9.8 \, m/s^2$, যেটা একটা $\rho = 3000 \, kg/m^3$ লোহার বলের ক্ষেত্রে প্রযোজ্য। এখন, এখান থেকে এই দাঁড়ায় যে $\alpha = 0.5 $। 

সত্যিকারে বলতে কি, কঠিন জ্যামিতি দিয়েও ঠিক একি সমাধানে আসা যায়। তবে আমরা আপাতত এটাতেই সন্তুষ্ট থাকি, পরে তা করা যাবে। 

আমরা বুদবুদের ক্ষেত্রে আগে বের করেছিলাম, 
\[ \alpha \rho_w a =  \rho_w g -  \rho g \] 
এখন আমাদের মনে আছে যে এই $\rho g$ একটা বুদবুদের ক্ষেত্রে অনেক কম (পানির তুলনায়), সুতরাং আমরা একে বাদ দিই, তাহলে, 
\[a  = \frac{ \rho_w g } {\alpha \rho_w } \qquad \rightarrow \qquad a = \frac{g}{\alpha} \]
Then we can just substitute the quantitative value of $\alpha$ (তারপর আমরা শুধু $\alpha$ এর সমীকরণ উপরে বসিয়ে দিয়ে পাই), 
\[a =  \frac{g}
{\frac{1}{\rho_0} \left( \frac{\rho_{\text{মেটাল}} g - \rho _0 g}{a_o} - \rho_{\text{মেটাল}} \right) 
}
\]
এবং মানের কথা বলতে গেলে আসে $a = 20 \, m/s^2$ । অর্থাৎ একটা পড়তে থাকা বস্তুর ত্বরণের ্দিগুন। খারাপ না, কিন্তু সত্যিকারে এটা আসলে একটু বেশি। কেন? আমরা সান্দ্রতা এবং এসব হিসাব করিনি, তবে সেটা করলে ত্বরণ আরো কম আসবে। এই ত্বরণের মান ঘন তরলের জন্যে ২ এর কাছাকাছিও হতে পারে। আবার কথা হলো বুদবুদ পুরোটা সময় ত্বরান্বিতও হতে পারে না, কারণ সে তার আগেই প্রায়ই Terminal Speed অর্জন করে ফেলে। বেশ কিছু জিনিসপত্র যোগ করলেই আমাদের উত্তর আরো গ্রহণযোগ্য হবে। 

\newpage
\textbf{একটা বুদবুদ এবং তার নাড়াচাড়া । } \\ 
{\tiny Part 3 of 3.} \\
আমরা দেখলাম একটা বুদবুদের নাড়াচাড়া মোটামোটি কেমন হতে পারে। তবে এবার একটা ছোট আলোচনা করব। 
একটা প্রশ্ন করি, 
\begin{center}
একটা বুদবুদের ভেতর বায়ুচাপ কেমন? বায়ুর ঘনত্ব কেমন?   
\end{center}
এটার উত্তর বের করতে আমাদের তেমন কষ্ট হবে না। 

আইডিয়াল গ্যাস সমীকরণ থেকে আমরা জানি, 
\[ pV = nRT \]
এখানে $n$ হচ্ছে কত মোল গ্যাস আছে এবং এক মোল গ্যসের ভর $M$ হলে এবং ভর $m$ হলে, $n = m/M$ । 
আর আমাদের মনেই আছে যে ঘনত্ব হলো $\rho = m/V$, তাই, 
\[ pV = \frac{m}{M} RT \quad \rightarrow \quad p = \frac  {m} {MV } RT \]
\[ p = \frac{\rho}{M} RT \] 
তাহলে এই সমীকরণ থেকে আমরা গ্যসের চাপ, তাম্পমাত্রা এবং ঘনত্ব কে একসাথে প্রকাশ করতে পারি। 

এখন আর্কিমিডিস পড়লে মনে থাকার কথা যে তরলের পৃষ্ঠ থেকে $h$ দুরুত্ব গভীরে থাকলে চাপ 
\[ p = h \rho_w g \]
এবং আমাদের গ্যসের সমীকরণে বসালে আসে, 
\[ h \rho_w g = \frac{ \rho}{M} RT \] 
এবং এখান থেকে বের করে নেয়া যায় ঘনত্ব কত। 
\[ \rho =  \frac{M h \rho_w g }{ RT }  \]
আচ্ছা বের তো করলাম এখানে একটু বিপদ আছে। $T$ বা এই তাপমাত্রা কে নিয়ে কী করা যায়? 

বুদবুদের ভেতরকার তাপমাত্রা তার ঘনত্বের উপর ভালই কাজ করে। তবে যদি বুদবুদ আস্তে আস্তে নাড়াচারা করে, বাহির থেকে যদি খুব বেশি কিছু না করা হয়ে, তাপ যদি না প্রয়োগ করা হয়, তাহলে $T$ মোটামোটি ধ্রুবকই থাকে। এর থেকে তখন ঘনত্ব বের করা কঠিন হয় না। 


\begin{tcolorbox}[breakable]
আমাদের এই লেখা এখানেই শেষ করা যায়, কিন্তু একটা ছোট প্রশ্ন করি, কোনো অঙ্ক নেই। 

বুদবুদের তো সবসময় উপরের দিকেই উঠবার কথা। কিন্তু একটা বিশেষ ক্ষেত্রে কিন্তু বুদবুদ ডুবতেও পারে। সেক্ষেত্রে পানি হয়তো কোনো কিছু একটা করে। 

\textbf{পানির আচরণ কেমন অথবা কী হলে একটা বুদবুদের পানিতে ডুবতে থাকা সম্ভব?} এই সমস্যাটা ধারনার উপর নির্ভর করে, তাই গাণিতিক প্রমাণের কোনো দরকারই নেই।    
\end{tcolorbox}
\
\
\\
\textbf{Reference:} Mechanics: Jaan Kalda: ``Study Guides for IPhO" \\
and, Oliver Lindstorm's Paper on Bubbles in Medium. 
\\
\
\\
\
\\
\begin{flushright}
\textit{Ahmed Saad Sabit} \\
\textit{06 August, 2020}
\end{flushright}



























\end{document}