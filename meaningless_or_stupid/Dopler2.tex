\documentclass[a4paper]{article}

%%\usepackage[T1]{fontenc}
%\usepackage{textcomp}
%\usepackage[dutch]{babel}
\usepackage{amsmath, amssymb}
\usepackage{tcolorbox}

% figure support
\usepackage{import}
\usepackage{xifthen}
\pdfminorversion=7
\usepackage{pdfpages}
\usepackage{transparent}
\usepackage{hyperref}
\title{The Doppler Effect: Part 2}
\author{Ahmed Saad Sabit }
\date{\today} 

\pdfsuppresswarningpagegroup=1

\begin{document}
    \maketitle
So, we will now find the doppler effect for sound, the \emph{Acoustic Doppler Effect. }

But at First, I will tell a very necessary \emph{``Claim''}. 
\begin{center}
    \textsc{The speed of sound in a medium is independent of the speed of the source.}
\end{center}
Why? I will tell the reason rigorously and you don't need to understand why. Just look at the material. 
%%%%%%%%%%%%%%%%
%%%%%%%%%%%%%%%%
%%%%%%%%%%%%%%%
\section{Why Speed of Sound is independent of source speed (And you don't need to understand this too much)}
We have the \emph{Continuity Equation for any matter }. 
\begin{equation}
\int_{S}^{}          \rho \vec{u} \cdot       \, 
\mathrm{d} \vec{S}
\end{equation} 
And then we can have the \emph{Euler Equation for Fluid}, 
\begin{equation}
- \frac{1}{\rho} \nabla p = \frac{ D \vec{u} }{ Dt } 
\end{equation} 
Now, the $Du/dt$ is the convective derivative. It is the acceleration of the fluid if the field function of the 
fluid is dependent on the position and time. 

If you look for the one dimentional case, and substitute $s = \frac{ \delta \rho }{ \rho }$, then, introducing the bulk modulus as, 
\[ B = - \frac{1}{V_0} \frac{ dp }{ dV }\]
we can finally come to, after solving the above two equation with given mathematical equality, 
\begin{equation}
\frac{\partial ^2 s}{\partial x^2 }= \frac{ \rho }{ B } \frac{\partial ^2 s}{\partial t^2} 
\end{equation}  
Which is same as any wave equation, 
\begin{equation}
\frac{\partial ^2 y}{\partial x ^2 } = \frac{1}{v^2} \frac{\partial ^2 y}{\partial t^2}
\end{equation} 
For this reason, the propagation speed of waves in any fluid medium is, 
\begin{equation}
v_s = \sqrt{\frac{ B }{ \rho }} 
\end{equation} 
Let me tell this in short what the hell has been done above, we start off with \emph{Mass conservation in Fluid Flow}
and then use the \emph{Euler's Equation}, that is kind of the Newtons Second Law for any fluid flow. Then introducing 
a mathematical substitute for density (that I called $s$ ), we find that the density of fluid is to oscillate, because this 
$s$ satisties the wave equation. But the wave equation has a term $v_s$ that is the propagation speed of wave in the fluid. 

This wave is just the oscillation of pressure in the fluid if the fluid is assumed to be air. And this is actually sound. 

This wave if has large amplitude, will sound high volume, and if has high frequency will sound very sharp. So, the only take away from this is, 
\begin{equation}
v_s = \sqrt{\frac{ B }{ \rho }}
\end{equation} 
Which is the speed of propagation of wave; or say; speed of sound in air.

But notice a fact, this speed \emph{DOES NOT} depend on the source speed. The air will be oscillating propagating a wave, but this speed of wave will not 
at all depend on the source speed. 

Or, think in this manner, all what is happening about sound is that the air is going back and foorth. And this back and forth going
will obviously depend how ``rigid'' the air is (bulk modulus) and how ``dense'' the air is. Theres no dependence on the source, because this 
wave is motion of ``medium'', not any random particle beiing emitted from a source. 

So, this claim is proved. Or if you don't like this, check the link in the last section. 

\section{Getting our head around moving sources and observers}
So what's up with moving sources and moving receivers? We will continue by following a very necessary understanding. 

I have already told that the speed of sound will be same no matter what the speed of the source is. But, if you have read the past paper (part 01), 
there is loophole in my diction. If speed of sound is same, then relative to whom? 

Guess, can it be relative to source? 

NO, NOT AT ALL, the speed of sound will be constant with \emph{respect to the Medium.} 

\begin{center}
    \textsc{The speed of sound in a medium is independent of the speed of the source, with respect to the Medium.}
\end{center}

And this is what we had assumed in the long derivation above. If there is wave propagation, then this wave will seem to look moving at a spee of $v_s$ to 
the resting Earth (relative to whom, the air is at rest). 
And someone moving through the sound wave propagation (towards the source), will certainly measure the wave speed to be more (because he is moving). 

If the source is moving, even he will measure different wave speed relative to him, because he is moving with respect to the medium and the wave 
speed is supposed to be constant relative to the medium, not source. 

So, if we draw the wave fronts of sound with respect to the medium, it will be perfectly spherical, without any dependence of the motion of the source. 

But, if the sound is continuously rings, then the moving source can cause a conical shape of many wavefronts to form. 

%figure put here, wave cone. 

\section{Doppler Effect }
Let us make this scenario as we did in the last part of the previous note (Part 01). The source, (actually emitter in the previous note) have a speed $v_{source}$ 
and the receiver (observer, or the wall from the previous note) have a speed $v_{observer}$, so, we can tell, if the speed is such that the 
receiver and source is coming closer, the source measures speed of source, 
\[ \mathbb{V} =| v_{source} + v_{receiver} | \]
If the situation is that the source and receiver are relatively moving away from each other, 
\[ \mathbb{V} = | v_{source} - v_{receiver} |  \]

You can visualize the situation in this sort of manner, draw a line that connects the source and receiver, 
if the line becomes smaller with time, then the distance between them is decreasing, if the line is expanding,
distance between is increasing. I might re copy past some old diagrams from previous note. 
% old diagram on relative position. 

Now in Doppler Effect, if we choose the Medium referance frame (which is the best to choose as the sound speed remans constatn),
we can have an anarchy of source and receivers moving randomly, but, we can immediately change to source frame or receiver frame, bring additional
required changes to the sound speed in these frame, and like so we can solve for the things we require. 

Now, we should recall, that the sound speed is the equation \begin{equation}
v_s = \lambda f
\end{equation} 
Or, we can also use this $f = \frac{1}{T}$ to tell that, \begin{equation}
v_s = \frac{ \lambda }{ T }
\end{equation} 
I will use the $v_s = \lambda f$ for procedure. 

Let we have the situation, 
\begin{tcolorbox}
    Speed of emitter/ source respect to the medium $= v_e$ \\
    Speed of receiver respect to the medium $= v_r$ \\
    The direction of motion of both source and receiver can be backwards or forwards. \\
    Medius is at rest respect to Earth, and we are on earth measuring things (as usual physics people next door.) \\
    Now let some sound of frequency $f_0$ (measured from source), speed $v$ in medium be emitted from source. Then, 

    \emph{What shall be the frequency of the sound relative to the observer?}
\end{tcolorbox} 
I will solve this in very short text. Because there is possibility that the motion cna be backwards or forwards, we can put a $ \pm $ 
in the speed sign. Let the sound propagate in the right direction (increaseing x) and increasing x be positive.

Relative to Medium or Earth, the speed of receiver is $v_r$ and speed of sound respect to the Earth is $v$. So, speed measured by the 
receiver of sound can be, 
\begin{equation}
v* = v \pm v_r
\end{equation} 
Please remeber how relative speed relation follows when changing from one frame to another (Part 1 of the Note). 

Now there is a varied speed in the receiver. But please be little cautious, kinematically we are more inclined to directly relate the source and receiver, 
but here the situation is quite different, the speed of sound remains constant in the medium frame. 

I would also like to clean some mess and make the procedure look more straightforward to think, so without making analysis of speed, let us analyze the wavelenghts.

In the source frame, the speed measured should be, 
\begin{equation}
\tilde{v} = v - (\pm v_e)
\end{equation} 
The negetive sign because, if the emitter is moving towards the direction the sound is moving, the speed has to be subtracted, and if the 
source moves in opposite direction to the wave speed, the speeds relative add. 

$\tilde{v}$ is speed measured by emitter and $v*$ is speed measured by source. And $v$ is measured by Earth (us.).
What will be the sound frequency measured by the receiver?
Let us begin inputting the $\lambda f = v$ in the equation of speed for receiver.
\begin{align}
    v* & = v \pm v_r \\
    f* \lambda* & = f \lambda \pm v_r 
\end{align} 
The stars are the things measured by the receiver. And un-starred are measured by us. What is the wavelength measured by the receiver and us? And also, what frequency do we measure? 

For the case of emitter, 
\begin{align}
    \tilde{v}& = v - (\pm v_e)  \\
    f_0 \tilde{\lambda} &= f \lambda - (\pm v_e) \qquad \text{Lots of Laugh, we have actually isolated the} f \lambda \text{at one part.} \\
    f \lambda &= f_0 \tilde{\lambda} + (\pm v_e) 
\end{align} 
Putting the $f \lambda$ in the equation $  f* \lambda*  = f \lambda \pm v_r$ finds us, 
\begin{align}
    f* \lambda* & = f \lambda \pm v_r \\
    f* \lambda* & =  f_0 \tilde{\lambda} + (\pm v_e) \pm v_r \\
    v \pm v_r &=  f_0 \tilde{\lambda} + (\pm v_e) \pm v_r \qquad \qquad \text{Using } v \pm v_r = v* = f* \lambda* \\
    \end{align}  























































\section{Some extra stuff}
\begin{enumerate}
    \item If you need more depth in low time, \url{https://physics.stackexchange.com/questions/237676/does-sound-waves-pick-up-the-speed-of-its-source}
    \item Concepts in Thermal Physics: Blundell.
\end{enumerate}
\end{document}