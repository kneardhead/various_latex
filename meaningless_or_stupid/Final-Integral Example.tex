\documentclass{article}
\usepackage{amsmath,amsthm}
\usepackage[left=2cm,right=2cm,top=2.8cm,bottom=2cm]{geometry}
\theoremstyle{definition}
\newtheorem{prob}{\textsf{Problem}}[section]
\theoremstyle{definition}
\newtheorem{sol}{\textsf{Solution}}[section]
\theoremstyle{definition}

\title{\textsf{Understanding how \LaTeX\  handles Integrals}}
\author{Ahmed Saad Sabit}
\date{22 Sept 2019}
\everymath{\displaystyle}
\begin{document}
\maketitle
\begin{abstract}
I try to write an integral with correct alignment in \LaTeX. We start doing the integration of a very rude mathematical expression often faced in calculations with \emph{Electric Fields}. The motivation behind the paper is trying to make a assorted discrete reference to deal with the integration of $\frac{1}{\sqrt{1 + x^2}}$ or such sort of using trigonometric aid.
\end{abstract}
\section{General}
\theoremstyle{definition}
\begin{prob}
 Solve the following integral $\int\frac{x}{(x^2+a^2)^{\frac{3}{2}}}\, \mathrm{d}x  $
\end{prob}
\begin{sol}
Our strategy would be to eliminate the things we aren't familliar with, in this case the integral with \emph{powers and powers above the bracket}. Hopefully, this simplification becomes very doable in case we put \emph{Trigonometric DIfferentials} in place of odd things.
\begin{align}
\int\frac{x}{(x^2+a^2)^ {\frac{3}{2}}} \, \mathrm{d}x \notag = \int\frac{a\,tan\theta}{ (a^2\, tan^2\theta + a^2)^ {\frac{3}{2}} } \, a\, sec^2\theta \,\mathrm{d}\theta
\end{align}
As we make an educated guess that taking a and x as a side of a right angled triangle with $\theta$ angle in between,
\begin{align}
x & = a \, tan\theta  \notag \\
 x^2 & = a^2 \, tan^2\theta  \notag \\
 \mathrm{d}x& = a \,sec^2 \theta\, \mathrm{d}\theta \notag
\end{align}
So we move to specifically chase what we sought of,\\
\begin{align}
\int \frac{a^2 \, tan\theta \,sec^2\theta}
    		{ (a^2)^{\frac{3}{2}} (tan^2\theta + 1)^{\frac{3}{2}} } \, \mathrm{d}\theta	\notag 
&=\int \frac{a^2 \, tan\theta \,sec^2\theta}
			{a^3 \, (sec^2\theta)^ {\frac{3}{2}} }	\, \mathrm{d}\theta	\notag \\
&= \frac{1}{a} \, \int \frac{tan\theta \, sec^2\theta}
								{sec^3\theta} \,\mathrm{d}\theta		\notag \\
&= \frac{1}{a} \, \int tan\theta \, cos\theta 	\, \mathrm{d}\theta \notag \\
%
&=\frac{1}{a} \, \int sin\theta \, \mathrm{d}\theta \notag	\\
&= - \frac{1}{a} \, cos\theta + C_0 \\
\intertext{By simply bringing back what we did to the x's and a's,}
&= - \frac{1}{\sqrt{x^2 + a^2}} + C_0
\end{align}
\end{sol}
That's how we barybash a stupid integral like a pro ! :)\\

\textsf{\textbf{The Idea's making this document}}\\
I had to struggle getting okay with the alignment rule. Its better not to clutter things but keep them broken in parts because the math environment don't really care about it. \\
I used the geometry package to keep things cool. Refer to this factory editor \LaTeX page for technical procedure's. Its to reduce the ridicoulous level of default spaces in the product pdf. \\
The "clearpage" declaration before starting the document reduces unwanted spaces.\\
Making the notag is somewhat a kind of relief. Keep in mind of the double slashes in align, dollar sign and "and" sign for normal stereotype maths.\\

\section{Formal Rules}
We have to recognize the algebraic structure and try to make the algebra look more like one the equation below. There are three general \textbf{Trig Derivatives} and \textbf{Anti - Derivatives}.\\
\textbf{\textsf{The Sin integration rule}}
\begin{equation}
\frac{d}{dx} \, \sin^{-1} x \, = \frac{1}{\sqrt{1 + x^2}}
\end{equation}
\[
\int  \frac{1}{\sqrt{1 + x^2}}\, \mathrm{d}x = \sin ^{-1}x + \, C
\] 
\textbf{\textsf{The Tangent integration rule}}
\begin{equation}
\frac{d}{dx} \tan ^{-1}x = \frac{1}{1 \, + \, x^2}
\end{equation} 
\[\int \frac{1}{1 \, + \, x^2} \, \mathrm{d}x = \tan ^{-1}x \, + \, C\]
\textbf{\textsf{The Tangent Coefficient rule}} \\
We can rather derive it from above. But it's the most common form we use the tangent rule.
\[\int \frac{1}{a^2 + x^2} \, \mathrm{d}x \,=\, \frac{1}{a} \tan ^{-1} (\frac{x}{a}) \,+\, C
\]
\section{In Action}

\begin{prob}
Solve : $ \int \frac{1}{9+x^2} \, \mathrm{d}x$
\end{prob}

\begin{sol}
We can see the $9 + x^2$ part make a nice match with the tan coefficient. It's not always necessary to have a perfect square in that position of 9, its something like 7, then write it as $a^2$ during the math and at the end deal with the a as a simple $\sqrt{7}$. This fact is also usable in very special cases if done properly, but not shown here.\\
So write the 9 as $3^2$ and we're done.
\[ \int \frac{1}{3^{2} + x^2} \, \mathrm{d} x \, = \, \frac{1}{3} \tan ^{-1}
 																									 (\frac{x}{3}) \, + \, C \]

\end{sol}



\begin{prob}
Solve : $ \int \frac{\sin ^{-1}x}{\sqrt{1 - x^2}} \, \mathrm{d} x $
\end{prob}

\begin{sol}
It looks scary, but whenever you see some symmetry, try if u-substitution works or not. We see the symmetry of the numerator and the denominator. 
\[ 
 \int \frac{\sin ^{-1}x}{\sqrt{1 - x^2}} \, \mathrm{d} x 
 \]
See that the derivative of $\sin^{-1} x$ is simply $\frac{1}{\sqrt{  1 - x^2      }}$, so
\[ u = \sin^{-1} x	\; \rightarrow \; du = \frac{1}{\sqrt{  1 - x^2      }} \, dx \]
\[\int u\, du \, = \, \frac{u^2}{2} \,+\, C \\
= \frac{(\sin ^{-1}x)^2}{2} \,+\, C\]
\end{sol}

\begin{prob}
Solve : $ \int \frac{x+9}
								{x^{2} + 9} \, \mathrm{d}x $
\end{prob}
\begin{sol}
I myself look forward to apply the idea myself. We have a 9 above that doesn't let us do the u-substitution and again we have an x above that don't let us to factor out anything, so the idea is something to make the previous ideas happen, we do it by breaking the numerator legally. We are able to write that
\[		\frac{x + 9}{x^{2} + 9}
		\;
		=
		\;
		\frac{x}{x^{2} + 9} 	+ 		\frac{9}{x^{2} + 9}
\]
That enables us to apply the two of our ideas separately, we need to the two integrations,
\[			\int	\frac{x}{x^{2} + 9} \, \mathrm{d} x	+
			\int	\frac{9}{x^{2} + 9} \, \mathrm{d}x  \]

For the first one we do the u-sub,because\textbf{ the derivative of the denominator is equal to the numerator}. 
\[ u = x^{2} + 9 \]
\[ du = 2x \, dx \]
\[ x \, dx = \frac{du}{2}
\]
Use this and we will have the general case, and the second one to deal is just the coefficient tan one
\[ \frac{1}{2}	\int	\frac{\mathrm{d}u }{u}
			\, + \,
	9 \int \frac{1}{ x^{2} + 9}\, \mathrm{d} x
    \]
    \[ = \, \frac{1}{2} ln(u) \, + \, 9\frac{1}{3} \tan ^{-1} (\frac{x}{3})\, +\, C	
\]
\end{sol}
		
		\begin{prob}
		Solve : $ \int \frac{1}
		{\sqrt{7 - x^{2}   }} \, \mathrm{d} x
		$\\
		And I rather leave this one for some quick practice
		\end{prob}
		
		\begin{prob}
		Solve for an indefinite integral : $ \int_{x=0} ^{x=\infty}
			\frac{\lambda}
			{2 \pi \epsilon_0} \, \cos\theta \, \frac{dx}{y^2 + x^2} $
			I think it wouldn't be wrong to do a definite integral though.
			\end{prob}

		\begin{prob}
Don't solve it, but look for the potential approaches that you think would work : $ \int \frac{x}
                                                                                                                                                                {	( x^2 + k^2) ^ {\frac{3}{2}} } \, \mathrm{d}x $
                                                                                                                                                                \end{prob}
                                                                                                                                                                
Okay, this was a whole lot of typing for a newone as me, but thanks to Patrick from the "Just math tutoring", the trigonometric integration part has been motivated by him and the problems are from his video tutorial.\\
DONE.
\end{document}

















