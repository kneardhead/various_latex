\documentclass[12pt,a4paper]{article}
\usepackage{amsmath}
\usepackage{hyperref}
\usepackage{amsfonts}
\usepackage{amssymb}
\usepackage{graphicx}
\usepackage[top = 2.8cm, bottom = 2.8 cm, right = 3 cm, left = 3 cm]{geometry}
\usepackage{color}
\usepackage[banglamainfont = Kalpurush, banglattfont= Siyam Rupali]{latexbangla}

\title{পদার্থবিজ্ঞানের প্রবলেম সলভিং এর মৌলিক আইডিয়া}
\author{Ahmed Saad Sabit}
\date{21 February 2020}


\begin{document}
\maketitle
\subsection{Introduction}

পদার্থবিজ্ঞানের কিছু সাধারণ প্রবলেম সল্ভ করার জন্য আমরা কতগুলো নিয়ম বা "আইডিয়া" সম্পর্কে আলোচনা করব। "আইডিয়া" বিষয়টার আইডিয়া প্রফেরসর যান কালডা থেকে পাই, তারই কিছু Universal idea (সব জায়গায় সমান ভাবে ব্যবহার করা যায়) এখানে দেখানো হলো। এসব কিছু জান কালডার করা সেরা কিছু সৃষ্টি, এখানে সবই তার করা।
\url{ahmedsabit02@gmail.com}

\section{Geometrical Solution: Extremum Problems}
জ্যামিতি দিয়ে মোটামুটি সব লম্বা অংকের ঝামেলা কমিয়ে ফেলা যায়। আমাদেরকে কোন extremal অথবা minimal\footnote{মানে সবচাইতে কম দূরত্ব বা সবচাইতে বেশি সময় বের করতে হয়- সে টাইপের সমস্যাগুলো।} টাইপ প্রবলেম দিলে আমরা কিছু না বুঝে একটা এক্সপ্রেশন লিখে ওটা ডিফারেনশিয়েট করা শুরু করে দিই। তারপর ধরে নিই যে এই ডিফারেনশিয়েট করার মান ০ । অর্থাৎ maxima-minima বের করার মত\footnote{ফার্মাট'স প্রিন্সিপল এর মত।}। এটা মাঝে মধ্যে কাজে লাগলেও কম্পিটিটিভ ফিসিক্সে এটা ভাল কোন মেথড না।\\
%
একটা ভাল প্রবলেম দিয়ে আমরা বোঝার চেষ্টা করি। \\
\\
\textbf{প্রবলেম ০১:}দুইটা বিমান একি উচ্চতায় $v_1=800km/hr$ এবং $v_2 = 600km/hr$ গতিতে উড়ে। তাদের "ট্রেজেক্টরি" একে অপরের "পারপেনডিকুলার" বা লম্ব। এবং সময়ের শুরুতে তাদের দুরত্ব $a=20km$, তাদের "ট্রেজেক্টরি" (মানে যে দুটা পথ/লাইন বরাবর বিমানগুলো চলে) এর ছেদ বিন্দু থেকে। দুটি বিমানের সবচাইতে কম সম্ভব দুরত্ব বের কর। \\
\\
তুমি বিমান দুটির দুরুত্ব সময়য়ের ফাংশন হিসেবে লেখে তারপর ডিফারেনশিইয়েট করে সবচাইতে কম দূরত্বের সময়টা বের করার চেষ্টা করতে পারো, আমি নিষেধ করছি না। তবে এক্ষেত্রে একটা ত্রিপল সাইজ খাতা নিতে হবে, অসম্ভব বড় হবে অংকটা। আর এটার কোনও বিশেষ দরকার ও নাই। আমরা একটা সাধারণ আইডিয়া বসাব।\\
\\
\fbox{
\parbox{\textwidth}{
\textbf{Idea 01:} Use Geometric solutions over hopelessly difficult or long problems, if it seems feasible.
}
} \\
\\ %
তার মানে অংক অস্বাভাবিক কঠিন অথবা লম্বা হয়ে গেলে জ্যমিতিতে প্রবলেমটা সমাধান করার চেষ্টা করতে হবে। এখানে আমাদেরকে বলা হয়েছে "সবচাইতে কম দুরুত্ব বের করতে।" আমরা তাই একদম ডাইরেক্টলি করবো। \\
\textbf{কমনসেন্স দ্বারা আমরা জানি যে একটা বিন্দু থেকে আরেকটা লাইনের মধ্যে সবচাইতে কম দুরুত্ব থাকে যখন আমরা বিন্দু থেকে লাইন পর্যন্ত একটা লম্ব আঁকব। অন্য যেকোনো লাইন লম্ব থেকে বেশি হবে। }
 \begin{figure}[hbtp]
 \centering
 \includegraphics[scale=1.3]{../User/sepal01.png}
 \end{figure}
তাহলে আমাদেরকে এই সমস্যায় একটা লাইন আনতে হবে, কেমনে আনব তাই হলো একটা ভাল সল্ভারের পরিচয়। জান কালডার একটা ভাল আইডিয়া বসিয়ে দিতে পারি। \\
\\
\fbox{
\parbox{\textwidth}{
\textbf{Kinematics Idea 01:} সবচাইতে সুবিধাজনক "ফ্রেমে অফ রেফারেন্স"টা ব্যবহার করো। বেশ কিছু চয়েস আছে এক্ষেত্রে, এবং প্রয়োজনে এক ফ্রেম থেকে আরেকটাতেও যাওয়া যেতে পারে। কিছু ভালো ফ্রেম হলো,\\
১। যেটাতে কোনো বস্তু স্থির আছে;\\
২। যেটাতে গতির "প্রোজেকশন" বা "কম্পোনেন্ট" কমে যায় (নাইলে শুন্য হয়ে যায়)।	\\
৩। যেটাতে গতি সিমেট্রিক বা প্রতিসম/দুটি দিকে সাঞ্জস্যপুর্ন।}}
\\
\\
তো আমরা এ প্রবলেমে যা করবো তা হলো যেকোনো একটা বিমানের ফ্রেমে চলে যাব, তাহলে মনে হবে যেন আরেকটা বিমান আমদের কাছে যেমন আসছে, তেমন সাইডেও সরে যাচ্ছে, নিজেদেরকে একটা বিন্দু হিসেবে কল্পনা করি, যা পাব তা আমাদের জেনারেল আইডিয়া আর পিথাগোরাস থিওরেম দিয়ে সল্ভ করে ফেলা যাবে। তবে যাই হোক, আমি কিন্তু  কোনও সলিউশন দিব না, মেলানোর জন্যে উত্তরটা দেখতে পারো।  \\
%আনসার দাও! 
আমরা উপরে যে আলোচনা করলাম, এ মেথডে কোনো এক্সট্রিমাল সমস্যা সলভ করতে হলে মূলত দুটো কাজ করতে হয়, \\
১) Problem Structure Exploit বা সমস্যার গঠনকে নাড়িয়েচাড়িয়ে নিতে হয় (উপরে আমরা ফ্রেম বদলালাম) । \\
২) Extremal Condition বা "সবচাইতে বেশি হওয়ার কারন/কন্ডিশন" জ্যমিতিকভাবে প্রকাশ করতে হয় (আমরা বিন্দু-লম্ব লাইন আইডিয়া যেমন ব্যবহার করলাম)।\\
৩) Conclude বা অংক করে শেষ করা। \\
আরেকটা কথা, একাজ কিন্তু ভেক্টর দিয়েও করা যায়!
%
তবে এটা যথেষ্ট না, কারণ এ আইডিইয়া কোণ এর সাথে কাজ করেনা, মাঝে মধ্যে আমাদের extremal angle খুঁজতে হয়, সেটার জন্যে আমরা একটা কৌণিক আইডিয়া ব্যবহার করবো।\\ 


\includegraphics[scale=.5]{../User/sepal0002.jpg}
\\
\\
\fbox{
\parbox{\textwidth}{
\textbf{যদি আমরা একটা ত্রিভুজ $\triangle ABD$ নিই যার $AB,BD$ ধ্রুবক থাকে এবং $AD$ কমবেশি হতে পারে, তাহলে $\angle DAB$  তখনই সবচেয়ে বেশি হবে যখন $\angle ADB = 90\deg$ হয়, অথবা $BD$ লাইনকে B কেন্দ্র করে ঘোরালে যে সার্কেল(বৃত্ত) তৈরি হয়, তার স্পর্শক A দ্বারা গেলে। }}}
\\
\\
%
আমাদের ঘরের ফিজিক্স অলিম্পিয়াডেই একবার এমন একটা প্রবলেম আসছিল, তখন ক্যালকুলাস এর প্রশ্নই উঠে না। \\
\begin{figure}[hbtp]
\centering
\includegraphics[width = 0.8 \textwidth]{../User/sepal03.png}
\caption{Illustration of the problem}
\end{figure}
\\
%
\textbf{প্রবলেম ০২: (NPhO 2014 Category B)} একটা $h$ উচ্চতার ল্যাম্পপোস্ট আছে, তার থেকে $d$ দূরে একটা রড আছে, যেটা মাটিতে হিঞ্জ দ্বারা লাগানো, মানে দরজার মতো কোণ বাড়ানো কমানো যায়। সেটা $r$ মিটার লম্বা, কোন কোণে রডটা থাকলে তার ছায়া সবাচাইতে লম্বা হবে? \\
\\
%
ছবিটা দিয়েই আমরা অনেকটা ধারনা পেয়ে যাই, যারা অন্ধ ক্যালকুলাস করবে, তাদের দ্বারা হবে না (আমার দ্বারাও হয়নি)। এখানে যে বিষয়টা আমরা খেয়াল করবো তা হলো আমরা রডের তৈরি করা Trajectory বা পথটি বিভিন্ন কোণ এর জন্যে বিশ্লেষণ করবো, এখানে যেহেতু ল্যাম্পপোস্ট (ছবিতে বাঁ' দিকের লম্বটি) রডের চাইতে লম্বা, তাই একদম হুবহু আমাদের extremal angle এর আইডিইয়া এখানে খাটবে না, তাও সেটা সরাসরি ব্যবহার করা যাবে। $F$ কোণ (ছবিতে) $90 \deg$ হলেই কিন্তু আমাদের প্রবলেম সল্ভড হয়ে যায়। তুমি দেখাতে পারবে যে $\angle F = 90 + \delta$ সামান্য বিচ্যুত হলে (কমে, নাইলে বাড়লে) ছায়ার দৈর্ঘ কমবে, তাই ৯০ ডিগ্রি হলো আমাদের maximum shadow length এর condition। এর থেকে জান কালডার আরেকটা আইডিয়া বলা যায়,\\
\\
\fbox{
\parbox{\textwidth}{
\textbf{Idea 02:} If $\varepsilon$ is some special value on which any variable is dependent on; and for extremely small increment or decrement $\delta$ the variable reduce both the times, then the special $\varepsilon$ is the maximal value for the variable.
}
}
\\
আরেকটু সহজভাবে বলতে গেলে, ধরি আমাদের কাছে একটা ফাংশন আছে $f(x)$। আমরা দুবার x এর মান পরিবর্তন করবো। একবার ছোট পরিমাণ $\delta$ বাড়াবো, আরেকবার কমাবো। তাহলে আমরা পাই $f(x + \delta), f(x - \delta)$; এখন যদি এমন হয় যে $f(x + \delta), f(x - \delta)$ দুটিই $f(x)$ এর চাইতে কম তাহলে অবশ্যই $f(x)$ হলো সবচাইতে বেশি (maximum possible)। \\

এবার এসব আইডিয়া কাজে লাগাবার জন্য আমরা কিছু প্রবলেম দেখব।

\includegraphics[width = 1\textwidth]{../User/sepal04.png} \\
\\
\textbf{প্রবলেম A2:} একটা সাধারণ পৃষ্ঠতল আছে, যা ভূমির সাথে $\theta$ উৎপন্ন করে। একটা ছোটো ব্লক আছে তার উপর (স্থির) যার ভর $m$ এবং $f = mg \cos \theta  \mu$ ঘর্ষন বল কাজ করছে। সবচাইতে কম সম্ভব কত বল দিয়ে ব্লককে তার স্থান থেকে সরানো যাবে?\footnote{Hint: যে সকল বল (Force) ব্লকের উপর কাজ করছে তার সবগুলোর resultant বা যোগফলটা বের করো, এখানে আমাদের আইডিয়ে কাজে লাগবে।}
\\
\\
\textbf{প্রবলেম A3:} আমরা কী আমাদের এই আইডিয়াটা Fermat's Principle এ ব্যবহার করে কিছু সহজ করে নিতে পারি?\footnote{Find the use of the maximal idea in Fermat's Principle, that light takes the trajectory which will take the least time (minimum time possible). জেনারালিটির জন্যে n অপটিকাল ইনডেক্স ধরে কাজ করো।}\\
\\
\textbf{প্রবলেম A4:} একটা H উচ্চতার টাওয়ার থেকে বিভিন্ন দিকে $v_0$ সমবেগে ছোটো বল ছোড়া হচ্ছে। সবচাইতে বেশি কত দূরত্ব টাওয়ারের পা থেকে বল অতিক্রম করতে পারে ?
\\
\\
এই প্রবলেমটা বলতে গেলে এই আইডিয়াটার জনক। এটা সলভ করা খুব জরুরি না, তবে চেষ্টা করে দেখতে ক্ষতি কী? মনে রাখবে এটা একটু কঠিন। প্যারাবলার হিসাব করে প্রবলেমটা সহজে করা যায়, আমরা এটা নিয়ে পরের সেকশনে আলোচনা করবো। আপাতত শুধু ভাব যে কী কী করলে সমস্যার ভাল উত্তর আসবে, সেসব লিখে ফেল, আবার সল্ভ করে ফেলিও না! মূলত এই সমস্যাটা দিয়েই আমরা একটা কনস্ট্রাকটিভ থিঙ্কিং করবো\footnote{গঠনমূলক চিন্তা-ভাবনা।}। 

\section{Geometrical Solution: The "Elegant" Approach }
এবার আমরা ভাল একটা আইডিয়াতে আসছি। আমাদের মুল উদ্দেশ্য হলো জ্যামিতি দিয়ে কাজ চালানো, আর কথা হলো প্রায় সবসময় জ্যামিতিক এপ্রোচ শর্ট আর সহজ হয়, তা ছাড়া জান কালডা ফিসিক্স কাপে সবসময় জ্যামিতিক এপ্রোচকে বেশি পছন্দ করেন। আমরা এবার এইটাই করবো। \\
যখন ডাইরেক্ট এলজেব্রা দিয়ে আমরা প্রবলেম লম্বা কষ্টকর গণিত দিয়ে সল্ভ করি, তখন সেটা Brute Force এপ্রোচ। আর জ্যামিতিকটা হলো Elegant এপ্রোচ। একটা খাঁটি প্রবলেম দিয়েই এই আইডিয়া বেশি ভাল করে বোঝা যায়।\\
\\
\fbox{
\parbox{\textwidth}{
\textbf{Idea 03:} Whenever there is a long problem, it surely accompanies a short but an Elegant solution; aided by Geometry or any other clever and creative insight}} \\
\\
\\
আমরা নিচের প্রব্লেমটার মাধ্যমে এই আইডিয়ার ধারনা নিব।\\
\\
\textbf{প্রবলেম ০৩:} একটা বলকে আমরা প্যরাবলিক ট্রেজেক্টরিতে $v_0$ গতিতে $\theta$ কোণে ছুড়ে মারি। অবশ্যই এই প্যরাবলার একটা Focus পয়েন্ট আছে, Launch point (যেখান থেকে আমরা বলে নিক্কেপ করি) থেকে ফোকাস এর দূরত্ব কত?\\
\\
Brute Force এপ্রোচে আমরা যা করবো তা হলো, ট্রাজেক্টরির একটা $y(x)$ ফাংশন লেখবও, তারপর ডাইরেক্ট্রিক্স\footnote{200 More Puzzling Problems in Physics এর Appendix এ ছোট কিন্তু ভালভাবে এসব বোঝানো হয়েছে, এমনে আমরা এত গভীর জ্যামিতি নিয়ে পরে ভাববো।} লাইন খুঁজে ফোকাস এর অবস্থান বের করে অংক কষব। মানে যেভাবেই হোক আমরা এলজেব্রা দিয়ে অঙ্কটা শেষ করার চেষ্টা করবো। করা যায়, তবে অবশ্যই এটা কোনো ভাল এপ্রচ না। দেখ কী বাজে পরিমাণের গণিত দরকার $y(x)$ এর জন্যে। 
\begin{equation}
y(x) = \tan \theta x 
+
\frac{g}{2(v_0 \cos \theta)^2} x^2
\end{equation}
আমরা জ্যামিতি দিয়ে ভাবার চেষ্টা করি, কঠিন জ্যামিতিতে যাব না। প্রথমত আমরা \textbf{ফোকাস} নিয়ে একটু ভাববো। আমরা লেন্স/দর্পন এসব থেকে জানি যে প্যারাবলিক আয়নায় একদম সঠিক ফোকাস পয়েন্ট তৈরি হয়, আবার আমরা আলোর ফোকাস এর দিকে প্রতিফলিত হবার ধারনাটা মাথায় রাখি, তাহলে ফোকাস বের করতে পারি। তাই জ্যামিতিক ভাবে ভাবতে গেলে আমরা প্রতিফলনের সূত্রকেও সমীকরণ বের করার কাজে লাগাতে পারি। তবে শুধু তখনই যখন আমরা প্যরাবলাকে একটা আয়না কল্পনা করে কাজ করবো। চিত্রটা খুব ভাল করে খেয়াল করো। ইচ্ছা করেই চিত্রটা একটু বড় করবো।\\
\begin{figure}[hbtp]
\centering
\includegraphics[width = 1\textwidth]{../User/parabola01.png}
\caption{প্রবলেম ০২}
\end{figure}
সত্যি এটা জরুরি কারণ এটা পদার্থবিজ্ঞানের Ballistic Range সমস্যার সবচাইতে ভাল সমাধান। আমরা লক্ষ্য করি যে {\color{red} Focus angle বা ফোকাস এর কোণ (ভূমির সাপেক্ষে)$\alpha$} এবং {\color{green} Launching Angle বা নিক্ষেপ কোণ $\theta$}। আমি প্রবলেমের মুল অংশগুলো এপ্রোচ করার জন্য দিয়ে দিচ্ছি। 
\begin{figure}[!ht]
\centering
\includegraphics[width = .6\textwidth]{../User/sepal05.png}
\caption{Additional to প্রবলেম ০৩}
\end{figure}
\begin{itemize}
\item নিক্ষেপ কোণ আর ফোকাস এর কোণ এর মধ্যে একটা বিশেষ সম্পর্ক আছে, কারণ আমরা দেখতে পাচ্ছি যে ফোকাস লাইন $F$ এর ভূমির উপরের ছায়া (x এক্সিস কম্পোনেন্ট) এর দুগুণ হলো বলের মোট রেঞ্জ (বল যে দুরুত্ব অতিক্রম করে)। আর আমরা তো এমনেই জানি যে $\theta$ এর উপর আমাদের রেঞ্জ নির্ভরশীল (বানান?)। রেঞ্জ আবার $v_0$ এর উপর নির্ভরশীল, অর্থ হলো আমাদের ফোকাস এর দূরত্ব গতির কোনো একরকম ফাংশন। 
\item উপরের কথাটা একটু অন্যভাবে চিন্তা করলে আসে যে আমরা যদি $\theta$ বাড়াই (উপরের দিকে করে বল নিক্ষেপ করি), তাহলে আমাদের বলের ট্র্যজেক্টরি (যে পথ দ্বারা বল যায়; প্যারাবলা) কিছুটা উপরের দিকে সরে যাবে, মানে সবমিলিয়ে আমাদের ট্রেজেক্টরি উঁচু পাহাড়ের মত হয়ে যাবে, মানে ফোকাস এর কোণও বাড়াবে ($\alpha$ বাড়াবে)।
\item চিত্র থেকে আমরা দেখতে পারি যে পুরো রেঞ্জের আধার সাথে $\sec \alpha$ গুন করলেই $F$ পাওয়া যাবে। তার জন্যে আমাদের $\alpha$ জানতে হবে, এবং প্রথম হিন্ট থেকে আমরা ভাবছি $\alpha$ আর $\theta$ কে আমরা একটা সম্পর্কে আনবো। এবং একটা সাধারণ ডায়াগ্রাম (চিত্র) আঁকলেই আমরা সে ক্লু পাচ্ছি। 
\item যেহেতু আমরা প্যারবলাআ নিয়ে কাজ করছি, তাহলে আমরা কল্পনা করি যে আমরা আসলে একটা প্যারাবলিক আয়না নিয়ে কাজ করছি। প্যারাবলিক আয়নায় ভূমির সাপেক্ষে লম্বভাবে আলো দিলে সেটা ফোকাস পয়েন্টে যাবে। ধরে নিই আমরা নিক্ষেপ করার স্থান এর পয়েন্টে এমন নিচ থেকে আসে আলো ফেললাম। আলোটা যে পথ ধরবে, সেটাই নিক্ষেপ স্থল থেক ফোকাস বরাবর লাইন। এর থেকে আমরা দৈর্ঘের কিছু বের করতে পারবো না, কিন্তু এর প্রতিফলন কোণের তত্ত্ব ব্যবহার করে $\theta$ আর $\alpha$ এর মধ্যে ভাল গাণিতিক সম্পর্ক বের করে ফেলতে পারব। 
\end{itemize}

আমরা চিত্রের রঙগুলো নিয়ে হিসাব রাখব।\footnote{যদি না কালার ব্লাইন্ড হই!}
\begin{itemize}
\item {\color{green} নিক্ষেপ স্থলের প্যারাবলার স্পর্শক সবুজ।}
\item {\color{blue} নিক্ষেপের সময়কার গতির ভেক্টর বেগুনি।\footnote{আমি বুঝতে পারছিনা কোন রঙটা আসলেই বেগুনি।}}
\item উপরের ডট করা লাইনটা হলো ডিয়ারেক্ট্রিক্স, অবশ্য এটা আমাদের দরকার নেই, GeoGebra তে এটা ছাড়া প্যারাবলা আঁকতে দেয় না।
\end{itemize}
উত্তরটা খুব সুন্দর।
\textbf{ Answer:} $F = \frac{v_0^2}{2g} $\\
\\
\textbf{প্রবলেম B1:} $F$ এর কোণ বিশেষভাবে নিক্ষেপ করা বলের সরবচ্চ (বানান?) রেঞ্জ নির্ধারন করে, কিভাবে? চিত্র এঁকেই বের করার চেষ্টা করো। \\
\\
\textbf{প্রবলেম A4:} একটা H উচ্চতার টাওয়ার থেকে বিভিন্ন দিকে $v_0$ সমবেগে ছোটো বল ছোড়া হচ্ছে। সবচাইতে বেশি কত দূরত্ব টাওয়ারের পা থেকে বল অতিক্রম করতে পারে ?\\
\\
আবার বলছি এ সমস্যা সমাধান করো না, এতদূর যা শিখেছি তা দিয়ে তোমার আইডিয়ায় নতুন কিছু এসেছে কিনা লিখে ফেল। 

%
%
\section{Changing the Problem Mechanism:\\With Known Quantities and Constraints}
আমরা পদার্থবিজ্ঞানের বই পড়ি আর একটা প্রবলেমের সমাধান নিয়ে ভাবলে হাবুডুবু খাই, অথচ আমরা কিন্তু জানি যে আমরা যে টপিক পড়েছি, মূলত সেখান থেকেই প্রবলেমটা দেওয়া হয়েছে। তবুও কেন আমরা প্রবলেমটে সল্ভ করতে পারছিনা?


আসল কথা হলো যিনি প্রবলেম সেটার (যে প্রবলেমটা বানিয়েছেন) তিনি আমরা এইমাত্র যে থিওরি পড়লাম, সেখান থেকে হুবহু একটা সিচুয়েশন তুলে দেননি, তিনি "Structure" টা পালটে দিয়েছেন, যাতে আমরা আমাদের সদ্য পড়া থিওরি অন্য জায়গায়ও কাজে লাগাতে পারি। এর জন্য আমাদেরকে এই নতুন পরিবেশ, বা প্রবলেমএর এনভাইরন্মেন্টের সাথে খাপ খাওয়াতে হবে। তবে যে থিওরি বসাতে হবে, তা সে পুরনো জিনিসই, একজন ভাল সল্ভার সেরকম প্রায় সব পরিবেশেই তার থিওরির আর্সেনাল\footnote{``The Arsenal of Ideas... - Jaan Kalda, ``Physics Solvers Mosaic - IPhO 2012" প্রবন্ধ থেকে।} বা তত্ত্বের ঝুলি থেকে সবখানে ফর্মুলা বসিয়ে প্রবলেম সল্ভ করতে পারে। সাথে লাগে থিওরির চাইতে বেশি পরিমাণ আইডিয়া, যাতে সে প্রবলেমর যথাযথ গাণিতিক বিশ্লেষণ করা যায়। 


এবার মুল কথা বলি, একটা প্রবলেম সল্ভ করতে হলে আমাদেরকে আমাদের জানা থিওরিই ব্যবহার করতে হবে। আমাদেরকে শুধু এটাই জানতে হবে কোথায় কী করতে হবে। সে প্রবলেমটাই কঠিন যে প্রবলেম এর পরিবেশ এর সাথে আমরে খাপ খাওয়াতে পারিনা। তখন আমাদেরকে সেই প্রবলেমকে নানা আইডিয়া দিয়ে দংশন/কামড় যাচ্ছেতা করে আমাদের একটা জানা পরিবেশে আনতে হবে। আমি একটা ইরদোভ প্রবলেম দিয়ে বোঝাই।\\
\\
%
\textbf{প্রবলেম ০৪:} একটা কণা, যার নাম $A$, একটা $R$ ব্যসার্ধের বৃত্তর মধ্য দিয়ে ঘোরে। বৃত্তের উপরেই একটি বিন্দু $O$ থেকে $A$ কণা পর্যন্ত একটা $\vec{r}$ ভেক্টর দ্বারা কণার অবস্থান দেখাই। যদি এই $\vec{r}$ ভেক্টর $O$ বিন্দুকে কেন্দ্র করে $\omega$ কৌণিক বেগে ঘোরে, তবে কণার গতি কত? এই কণার ত্বরণ এর মান ও দিক নির্ণয় করো।\\
%
\begin{figure}[hbtp]
\centering
\includegraphics[scale = 0.45]{../User/sepal06.png}
\end{figure}
এখানে কৌণিক বেগ কিন্তু $\vec{r}$ এর, কণার নয়! \\
ভেবে দেখ, যদি আমরা পদার্থবিজ্ঞানের কোন বই পড়ি, তবে পড়ি যে একটা ঘুর্ননশিল (বানান?) বস্তু সম্পর্কে, কখনো পড়ি না যে একটা ভেক্ট্রএর ঘুর্নন। এখানেই তত্ত্বকে কাজে লাগিয়ে আমাদেরকে এটার পরিবেশের সাথে খাপ খাওয়াতে হবে। আমাদের ব্যবহার করতে হবে আমাদেরই কিছু পুরান বন্ধুকে, জ্যামিতি!\\
আমরা ধাপে ধাপে এই সমস্যার একটা সুন্দর সমাধান নিয়ে ভাবি। আমরা একটু গুছিয়ে কাজ করি,
\begin{itemize}
\item \textbf{Target Parameter: কণার গতি এবং ত্বরণ।}\footnote{আমরা যা খুঁজছি।}
\item \textbf{আমরা জানি,} একটা কণার কৌণিক বেগ কত হলে তার গতি কত। 
\item \textbf{আমরা জানি,} $\vec{r}$ যদি সময়ের সাথে সাথে দিক পরিবর্তন করে, তাহলে কণার সাথে বৃত্তের কেন্দ্রের লাইনের কোণেরও পরিবর্তন ঘটে।(বিভিন্ন স্থানে কণাকে চিত্রের মত আঁকলে পরিষ্কার হয় ব্যাপারটা)।
\item \textbf{আমরা জানি না,} $\vec{r}$ ঘুরলে কণার কী স্থান পরিবর্তন হবে, মানে কণার গতি ধ্রুবক থাকবে না কমবে বাড়বে সে সম্পর্কে আমাদের ধারনা নেই। এটাই আমাদের কে সমস্যার সমাধানের জন্যে ঘাটতে হবে।
\item \textbf{আমরা জানি,} এই বুকলেটে আগেই জ্যামিতির গুণগান গাওয়া হয়েছে!
\end{itemize}
ছবিটা একটু মনোযোগ দিয়ে দেখলে মাথায় আসে যে আমরা আমাদের $R$ নিয়ে আগে অনেক অংক করা হয়েছে। তবে এই $r$ লাইনটা আমরা জীবনে দেখিনি। আমাদের  Target Parameter  $R$ সম্পর্কিত, তবে আমাদের দেওয়া হয়েছে $r$ এর তথ্য। তো আমাদের কমনসেন্স বলে ভেক্টরটার সাথেই বৃত্তের কেন্দ্রের গাণিতিক সম্পর্ক বানাতে হবে। \\
সম্পর্ক আমরা এভাবে বানাই, \\
১) বৃত্তের কেন্দ্রকে আমরা $C$ ডাকি, যদিও ছবিতে তা দেখা যাচ্ছে না। এবার আমরা $OA$ এবং $OC$ লাইন আঁকি। $OC$ লাইন কে বাড়িয়ে সোজা বৃত্তের ডান দিকে নিয়ে যাই, তাহলে আরেকটা ব্যাসার্ধ $OX$ তৈরি হয়। \emph{তাহলে আমরা $\angle XCA$ সময়ের সাথে এর পরিবর্তনের মান $\frac{d \angle XCA}{dt}$ খুঁজছি।}\\
২)$OA$ এবং $OC$; দূটি লাইনই সমান, কারণ তারা একি বৃত্তের ব্যাসার্ধ।  সুতরাং, $\triangle COA$ সমদ্বিবাহু ত্রিভুজ (Isosceles Triangle, যার দুটি বাহু সমান থাকে), তাই $\angle COA = \angle CAO$ । একটা গুরুত্বপুর্ন সম্পর্ক তৈরি হলো। এখানে একটু মনোযোগ দিলেই বোঝা যাবে যে,
 \[ \frac{d \angle COA}{dt} = \frac{d \angle CAO}{dt}= \omega\]
৩) জ্যামিতি থেকে আমরা জানি যে, একটি ত্রিভুজের ভেতরকার দুটি কোণের যোগফল হল ত্রিতিও কোণের বাহ্যিক কোণের সমান। Sum of two internal angle is equal to the exterior angle to the third internal one. সুতরাং,
\[ \angle XCA = \angle CAO + \angle COA = 2 \angle COA \]
সবমিলিয়ে,
 \[ \frac{d \angle XCA}{dt} =
 2\frac{d \angle COA}{dt} = 2 \omega \]
 
 আমরা একটা সরল উত্তর পেলাম। কণা ঘুরছে $2\omega$ কৌণিক বেগে। এর থেকে বাকি উত্তর পেয়ে যাই। \\
এখানে আমরা আমাদের আইডিয়াটা প্রকাশ করতে পারি। \\
\\
\fbox{
\parbox{\textwidth}{
\textbf{Idea 04:} Take measures and legal steps so that the problem can be structured into something that is known}
}\\
\\
তবে শুধু এতদূরই নয়, আরো নানা ভাবে এই আইডিয়া ব্যবহার করে কঠিন প্রবলেম ভেদ করা যায়। ডেভিড মরিনের এই প্রবলেম এখানে ভাল এক্সাম্পল।\\
\\
\textbf{প্রবলেম ০৫:} একটি চার্জিত কিউব আছে, এর মাঝখানে (একদম কেন্দ্রে) তড়িৎ বিভব $\phi$ এবং এর দৈর্ঘ্য $L$। বের করো এই কিউবের কোণায় বিভব কত। \\
\\
কোনো দরকার নাই ক্যালকুলাস করে উত্তর বের করার কারণ আমাদের ইতিমধ্যে বলে দিয়েছে যে কিউবের কেন্দ্রে বিভব $\phi$। যেহেতু এটাই আমাদের একমাত্র প্রবলেম হিন্ট, তাই আমরা চেষ্টা করবো এটাই কাজে লাগানোর। \\
আচ্ছা, আমরা আমাদের পুর্বে আলোচিত আইডিয়াতেই ফিরে যাই। এমন কিছু একটা করতে হবে যাতে জানা জিনিসপাতি কে পরিবর্তন করে এখানে বসিয়ে দিয়ে সফলভাবে এই প্রবলেম সল্ভ করা যায়। তাহলে ভেবে দেখি, একটা কিউবের কোণার সাথে মাঝখনের কি সম্পর্ক?\\
ভেবে দেখলে সত্যি মাথা ঘুরাবে, এই প্রবলেমে কাজে লাগে এমন কোন কিছুই আমরা ভেবে পাই না। কেন্দ্রের বিন্দুর বিভব আর কোণায় বিভব সরাসরি মেলানো পরিশ্রম এর ব্যপার। \\
তো আমাদের জোর করে এই কোণার অংশকে কেন্দ্র বানিয়ে দিলে কী হয়? আসলেই হয়। ধরে নাও যে আমরা ৮ টা এমন কিউব একটা $2 \times 2$ রুবিক'স কিউব এর মত করে বসালাম। তাহলে বিভব ৮ গুন বাড়ল। কীসের ৮ গুন? কিউবের কোণার বিভবের। এবার এটা ধরে কয়েকটা অনুপাত নিলে অনেক সহজেই প্রবলেম সল্ভড। এই ৮ গুন কোণার হিসাবকে কেন্দ্রের বিভবের সাথে মিলিয়ে নিতে হবে। 



\section{Directly finding the Target Variable}
বড় অংশের পদার্থবিজ্ঞান প্রবলেম এই ধরনের উত্তর খোজে, যেমন "সর্বশেষ গতি বের করো", "সর্বমোট বিভব কত?"। কিছু আছে যেমন উত্তরকে $\beta \frac{x^2}{m}$ করে প্রকাশ করা গেলে $\beta$ এর মান কত। আরো আছে, তবে এবার আমরা প্রথমটার মত প্রবলেমের একটা সাধারণ তবে কাজে দেয় এমন নিয়ম নিয়ে ভাবব। \\
জান কালডার কাইনেমেটিক্স আইডিয়াটা, \\
\fbox{
\parbox{\textwidth}{
\textbf{Idea 05: }Take the required variable as unknown and find some equations so that the required variable can be extracted. Assume required quantities if absent, they should vanish as the math is done.}}\\
যদি আমাদের এমন হয় যে বের করতে হবে $x$, তাহলে আমরা প্রবলেম এর দরকার অনুযায়ী কয়েকটি সমীকরণ লেখব এবং চেষ্টা করবো $x$ কে একদিকে রেখে উত্তরে আসার। প্রবলেম এ দেওয়া ভ্যারিয়েবল ছাড়া কিন্তু বাহির থেকে কিছু আনা আইনসম্মত নয়! যেমন প্রবলেম স্টেটমেন্টে $m$ দেওয়া নেই, তবে উত্তরে $m$ এলো, তাহলে কিন্তু উত্তর সঠিক নয়! প্রায় সবসময় এমন খুব প্রয়োজনীয় ভ্যারিয়েবল ধরে নিতে হয় এবং অংক করতে করতে তা বাদ পড়ে যায়। \\
অস্বাভাবিক কিছু প্রবলেমে কী খুঁজতে হবে তা আলাদা করে গাণিতিকভাবে বের করাই প্রায়ই একমাত্র উপায় হয়ে যায়। তখন ঠিক করতে হয়, যে আমরা এই এই ভ্যারিয়াবল বের করব, তখন সমীকরণ বের করে সেসব বের করার উপায় খোঁজাই মুল কাজ হয়ে উঠে। এসব অজানা ভ্যারিএবল হলো Target Variable।
কিন্তু মনে রাখতে হবে,\textbf{যিনি প্রব্লেমটি বানিয়েছেন তিনি আমার তোমার চাইতে বহুগুণ চালাক। তাই প্রায় সবসময় একটা সমস্যা সমাধানে যে জিনিসটা সবচাইতে প্রয়োজন, তা কখনো বলাই হবে না। তাই ধরে নাও যে প্রাসঙ্গিক কিছু একটা নিশ্চিত আছে যা সমস্যায় মোটেও বলা নেই তবে সমস্যার খাতিরে তাকে সামনে আনা ছাড়া আমাদের আর কোনো উপায় নেই।} আর কিছু সমস্যার প্রবলেম টেক্সট ছোটো হলেও কিন্তু সমাধান প্রায়ই বিশাল আকারের হয়, তাই যতক্ষণ তুমি নিশ্চিত যে তুমি মোটামটি সরল পথে আছো, ততক্ষণ সমাধানের খোঁজ চালাতে থাকো।। পরের সমস্যায় এমতাবস্থার কিছুটা স্বাদ পাওয়া যায়। \\
\\
\textbf{প্রবলেম ০০:} একটা কনটেইনারের আয়তন (volume) $V = 20l$ এবং এর মধ্যে হাইড্রোজেন এবং হিলিয়াম রাখা হয়েছে। কনটেইনারের তাপমাত্রা $T= 293k$ এবং চাপ $p = 1.03 \times 10^5 Pa$। দেখা গেলো যে কন্টেইনারে $m = 5.0gm$ ভরের গ্যাস আছে, তাহলে তাতে হিলিয়াম এবং হাইড্রোজেনের অনুপাত কতো?\\
\\
বের করতে হবে $\frac{m_1}{m_2}$ ।\\
যা যা আছে, তা থেকে যত বেশি সম্ভব কাজে লাগানো যায় এমন সমীকরণ খুঁজে নিতে হবে। \\  
আমাদের শুরু করতে হবে খাতার উপর বড় করে $pV = nRT$ লিখে। আমরা সহজে কল্পনা করতে পারি যে কন্টেইনারে $n = n_1 + n_2$ মোল গ্যাস আছে, ধরি $n_1,n_2$ আমাদের আলোচ্য দুটি গ্যাস। এঁকেই আমরা $ n = \frac{m_1}{M_1} + \frac{m_2}{M_2} $ লিখতে পারি, যেখানে $m_1,m_2$ ও $M_1,M_2$ হলো ভর এবং এক মোলের ভর। মোলের ভর কিন্তু অজানা নয়!\\
আমরা কিন্তু $n$ জানি, কারণ $\frac{pV}{RT} = n$, আর চালাকি করে ভ্যারিয়েবল কমাবার লক্ষ্যে $m_1 = m - m_2$ ব্যবহার করতে পারি। \\
আমারদের কাজ তাহলে এমন দাঁড়ালো যে, 
\[ n = \frac{m - m_2}{M_1} + \frac{m_2}{M_2} \]
এর থেকে দাড় করানো যায়,
\[ m_2 = \frac{nM_1M_2 - mM_2}{M_1 - M_2} \]
আবার এও দেখানো যাবে যে,
\[ m_1 = \frac{mM_1 - nM_1M_2}{M_1 - M_2} \]
এবার উপরের $m_1, m_2$ কে ভাগ দিয়ে $n$ এর আসল মান বসিয়ে উত্তরে আসা যায়, এই যে,
\[\frac{m_1}{m_2} = \frac{M_1 (mRT - pVM_2)}{M_2(pVM_1 - mRT)} \]
এমনও আসতে পারে যে, 
\[\frac{m_1}{m_2} = \frac{m(M_1 - M_2)}{\frac{pV}{RT}m_1m_2 - mM_2} - 1 \]
দুটিই একি জিনিস। খেয়াল করলে দেখবে যে তাপগতিবিদ্যার সমস্যা হলেও আমরা একবারও তাপগতি সম্পর্কিত কিছু উচ্চারণ করিনি, এক টানে সমীকরণ মিলিয়ে গেছি!

২০১৯ সালের ফিসিক্স কাপের এক অসম্ভব কঠিন সমস্যাই এই আইডিয়ের পূর্ণ ব্যবহার করে। এখন যে প্রবলেম সম্পর্কে আলোচনায় যাব তা নিতান্ত আগ্রহের জন্যে, আপাতত এটা বোঝা খুব জরুরি না, তবে এই সমস্যাটার সমাধানে মারাত্মক বুদ্ধি আর ক্রিয়েটিভ টেকনিক আছে বলে এটা একটু ঘেটে দেখব। \\
\\
\textbf{Physic Cup-01 2019:} একটা বস্তুকে পানির মধ্যে দিয়ে ঠেলে নিয়ে গেলে অনুভূত হয় যেন তার ভর বেড়ে গেছে। এই বেড়ে যাওয়া ভরকে আমরা \emph{added mass} বলবো, যখন এই ভর আর বস্তুর আসল ভরের যোগফল হলো \emph{effective mass}। \emph{Added mass} বস্তুর গঠনের উপর নির্ভরশীল। 

ধরে নাও এই বস্তুর আয়তন $V$ এবং এর Polarizability (বাংলা?) হলো $\alpha$ (মানে $\vec{E}$ বিদ্যুৎ ক্ষেত্রে এর ডাইপোল মোমেন্ট $\vec{P} = \alpha \vec{E}$), আর এই বস্তু কে homogenous\footnote{independent of position- স্থান নিরপেক্ষ} বিদ্যুৎ ক্ষেত্রে রাখলে এর মধ্যকার বিদ্যুতও homogenous হয়। 

ধরে নাও $\rho$ ঘনত্বের পানির মধ্য দিয়ে এই বস্তুকে ঠেলে নিয়ে যাওয়া হচ্ছে, তাহলে বের করো যে শুরুর দিকে বস্তুর added mass বা অতিরিক্ত ভরের মান কতো। তোমার উত্তরকে $V, \rho, \alpha$ এবং কিছু মৌলিক ধ্রুবক দিয়ে প্রকাশ করো। \\
\\
Jargon Alert!\footnote{সাবধান! একটু কঠিন অংশ!}
প্রথমত এখানে added mass হলো বস্তু শুরুর দিকে যতটুকু অতিরিক্ত পানিকে তার সাথে ঠেলে নিয়ে যায়, কিন্তু এই পানি কিন্তু কিছুটা প্রবাহতে থাকে বলে ঠেলে নিয়ে যাওয়া পানির গতি বস্তুর গতির চাইতে কিছুটা আলাদা হবে। এখন আমাদেরকে বলা হলো এই অতিরিক্ত পানির ভর বের করতে। প্রবলেমটা সল্ভ করতে হলে অন্তত কিছু খুঁজতে হবে, কেমনে এই added mass কে আমরা সাধারণ অংকে এনে প্রকাশ করবো?\\
এইটা কঠিন হলেও অসম্ভব কোনো সমস্যা না, পরে এটার মুল সমাধান নিয়ে ভাবা যাবে, কারণ এই পুরো বুকলেট হলো সাধারণ ডিফিকাল্টির প্রবলেম সল্ভ করার আইডিয়ায়র উপর, এতো কঠিন সমস্যা নিয়ে আমাদের অনেক বেশি জিনিসপত্র নিয়ে ঘাটাঘাটি করতে হবে যা বর্তমানে আমাদের লক্ষ্য নয়। \\
এই মারাত্মক সমস্যার মারাত্মক সমাধান করেছিল ব্রিটেনের থমাস ফোস্টার (জোড়ে হাততালি!), জান কালডা তার সমাধানকে সেরা সমাধানের রিকগনিশন (বাংলা?) (স্বীকৃতি) দিয়েছিলেন। তার উত্তরের আইডিয়াটাই আমরা ছোট করে দেখব।\\
ফোস্টার আগে ধরে নেয় যে $F$ বল $x$ অক্ষ বরাবর দেওয়া হয় বলকে, \textit{তখন নিউটনের দ্বিতীও সূত্র দ্বারা},
\[ m = \text{Mass of object} \]
%
\[ \Delta m = m_{effective} - m = \text{Added mass} \]
%
\[ F = \frac{d P_x}{dt} = m \frac{d u_x}{dt} + \frac{d}{dt} 
\left( \rho \int_{fluid} v_x(\vec{r}, t)\, dV \right) 
=
m_{effective} \frac{d u_x}{dt} \]

এখানে সর্বমোট ভরবেগের হিসাব নেওয়া হলো, যেখানে পাশের বড় ইন্টিগ্রালটাই অতিরিক্ত পানির ভরবেগ (একটু মনোযোগ দিয়ে বোঝার চেষ্টা করে দেখ)। সুতরাং এখানে থমাস পেয়ে যায় যে তাকে এই বিশেষ ইন্টিগ্রাল সমাধান করতে হবে, যাতে এই অতিরিক্ত ভর বের করা যায়। 
\[ \Delta m u_x = \left( \rho \int_{fluid} v_x(\vec{r}, t) \,dV \right) \]
$u_x$ হলো বস্তুর গতি আর $v_x$ পানির প্রবাহের যা স্থান এবং সময়ের ফাংশন, থমাসকে এই ইন্টিগ্রাল সমাধান করতে হবে যাতে Target Variable $\Delta m$ বের করা যায়।\\
সে আস্তে আস্তে পানির প্রবাহের একটা অবকাঠামো দাড় করায়, দেখে যে ম্যাক্সয়েল এর সমীকরণের সাথে পানির প্রবাহের সমীকরণ মিলে (যদি বস্তুর আপেক্ষিক ফ্রেমে হিসাব করা হয়), তারপর এই পানির গতি কে ধরে নেয় যে এই পানির প্রবাহ হল অনেকটা বিদ্যুৎ ক্ষেত্রের মতো, তখন সে একটা আস্বাভাবিক অনুমান নেয় যে 
\[
\vec{v'} (\vec{r} , t) = c \, \vec{E} (\vec{r}, t) \]
পানির প্রবাহ ও বিদ্যুৎ ক্ষেত্রের একটা যোগসূত্র, এর দ্বারা সে প্রবলেমে দেওয়া সবকিছুই কাজে লাগায়, অনুপাত নিয়ে $c$ দূর করে আর একটা উত্তরে আসে। মাঝখানে ডাইলেক্ট্রিকের জটিল অংশ সমাধান করেই অংক শেষ হয়। উত্তরটা সুন্দর, \\
\textbf{Answer:} $ \Delta m = \frac{\rho V}{\frac{\alpha}{\epsilon_0V}-1}$\\
\\
আমি থমাসের অফিশিয়াল সলিউশনের দুটি পাতার ছবি আইডিয়ার এবং এপ্রচ এর স্বার্থে বসিয়ে দেব। কী করে পানির হিসাব কে ম্যাক্সওইয়েল এ নিয়ে ফেলা হয়েছে তা আমাকেও বিস্মিত করে! একটু খেয়াল করো, থমাস ফোস্টার ``f" কে ``s" এর মত লিখে।  \\
%
\includegraphics[scale=1]{../User/salvo02.png} 
\newpage
\includegraphics[scale=1]{../User/salvo01.png} 
%
%
\section{The Art of Constraints}
Belikov এর General Methods in Solving Physics Problems এর খুব ক্রিয়েটিভ একটা আইডিয়া constraint putting আর সাথে constraint lifting। সোজা কথায় প্রবলেম খুব কঠিন হলে আমরা সেই প্রবলেমের সহজ ভার্সন সমাধান করে তারপর আসল সমাধানে মনোযোগ দিই। আগে আইডিয়াটা পড়ে একটা সুন্দর এক্সাম্পল দ্বারা আমরা আইডিয়াটি বোঝার চেষ্টা করি। 
\\
\\
\fbox{
\parbox{\textwidth}{
\textbf{Idea 06:} If required, put constraints in the problem in order to make it simple and come into a feasible solution, then lifting the constraints one by one come to the real solution. We do this so that the problem is solvable by \textbf{Idea 04}.}}
\\
\\
মুল কথা সমস্যাকে সহজ করে সমাধার করাই এবার আমাদের লক্ষ্য। \\
\\
\includegraphics[width=\textwidth]{../User/sepal07.png} 
\textbf{প্রবলেম ০৬:} ছবির মত সিস্টেমে আমরা ভরগুলিকে ঝুলিয়ে দিয়েছি, এখন বের করো সুতোর উপর টেনশন বল (বাংলা?) কত। $f_1,f_2,f_3$ হচ্ছে ঘর্শণ এর সহগ\footnote{Friction Coefficient.}।\\
\\
এখন এ সমস্যা অতিরিক্ত কঠিন হবে যদি আমাদের কে এই ওয়েজ (যার উপর এই দুটি ভর বসে আছে) ত্বরান্বিত হয় (অর্থাৎ সরে যায়)। তাহলে আমাদের উত্তর বের করতে আরেকটু বেগ পেতে হবে। আপাতত আমরা ধরে নিই যে ওয়েজ মাটির সাথে শক্ত করে লাগানো এবং তা নড়বে না। সুতরাং $f_3 = \infty$ এবং $m_3$ নিয়ে আপাতত আমাদের আগ্রহ নেই। পরে আরো সহজ কৌশলে আসল প্রবলেম আমরা সমাধান করবো। 


প্রবলেমকে চালাক ব্যাক্তিগন এক মাইরে করে ফেলবে, আমরা বরং আমাদের আইডিয়ার সাথে লেগে থাকি। আমরা কিছু constraint বসিয়ে এই প্রবলেমকে আমাদের জানা একটা প্রবলেমের সাথে মিলিয়ে নেব। \textbf{Idea 04} কে একটু স্মরণ করি। একটা খাতা নিয়ে  সমীকরণগুলো সমাধান করলে খুব ভাল হয়। \\
\\
\textbf{Constraint 01:} ধরে নিই, $f_1 = 0$, $ \alpha_1 = 0$, $ \alpha_2 = 90\deg$ ।\\ 
তাহলে আমাদেরকে শুধু দুটি সমীকরণ মেলাতে হবে, 
\[ m_1 a_{1x} = F_t \]
\[ m_2 a_{2y} = m_2g - F_t \]
\[ a_{2y} = a_{1x} \]
প্রথম দুটো সমীকরণ আসে আমাদের বলের হিসাব দিয়ে, শেষেরটি আমরা পাই এই ধারনা থেকে যে দুটির ত্বরণ সমান না হলে এই সুতো ছিঁড়ে যাবে। \\
\begin{figure}
\includegraphics[width=0.6\textwidth]{../User/sepal08.png}
\centering
\caption{Constraint 01}
\end{figure}\\
এর থেকে আমরা উত্তর পাই যে,
\[ a = \frac{m_2}{m_1 + m_2} g \]
সমীকরণ থেকে $F_t$ কে বের করলে,
\[F_t = \frac{m_1m_2}{m_1 + m_2} g \]
এটা আমাদের প্রথম কন্সট্রেইন্ট এর উত্তর। আরো কিছু constraint তুলে দিয়ে অংক করে দেখি। \\
\\
\textbf{Constraint 02:} এবার ধরে নিই, $f_1 \neq 0$, $ \alpha_1 = 0$, $ \alpha_2 = 90\deg$, মানে আগের হিসাব থেকে $f_1 = 0$ তুলে দিয়ে, 
\[ m_1 a = F_t - f_1 m_1 g \]
\[ m_2 g - F_t = m_2 a \]
প্রথমে $a$ বের করে আমরা দেখাতে পারি যে,
\[ F_t =  \frac{m_1m_2 (1 + f_1) }{m_1 + m_2} g \]
আরো কন্সট্রেইন্ট তুলে দিই। \\
\\
\textbf{Constraint 03:} ধরে নিই, এবার $\alpha_1 \neq 0$, তাহলে আমাদের সমীকরণ দাঁড়ায়, 
\[ m_1 a = F_t - m_1 g \sin \alpha_1 - f_1 m_1 g \cos \alpha_1 \]
\[ m_2 a = m_2g - F_t \]
\begin{figure}[hbtp]
\centering
\includegraphics[width=0.7\textwidth]{../User/sepal09.png}
\caption{Constraint 03}
\end{figure}
আগের নিয়মে গেলে আমরা (এবার একটু বেশি কষ্ট করে) পাই, %সাইনে ভুল আছে মনে হয় 
\[ F_t = \frac{m_1 m_2 (1 + f_1 m_1 \cos \alpha_1 + \sin \alpha_1)}
  {m_1 + m_2} g \]
আমাদের এই উত্তরের সাথে প্রথম কন্সট্রেইন্টের মিল আছে একরকম, 
\[ F_t = \frac{m_1 m_2 }{m_1 + m_2} (1 + f_1 m_1 \cos \alpha_1 + \sin \alpha_1) g \]
এই সমীকরণকে এভাবে প্রকাশ করলে, যদি $(1 + f_1 m_1 \cos \alpha_1 + \sin \alpha_1) = \varsigma$ হয়,
\[ F_t = \frac{m_1 m_2 }{m_1 + m_2} \varsigma g \]
\[ F_t =  F_{t0} \varsigma \]
তাই $\varsigma$ হল এমন এক সহগ যা নির্ধারন করে দিচ্ছে $F_t$ ওয়েজের গঠন (structure) এর উপর কেমনে নির্ভর করবে। \\
\\
%
%
\textbf{Constraint 04:} রীতিমত সব তুলে দিয়ে, $\alpha_2 \neq 90 \deg, f_2 \neq 0$ সহ,
\[ m_1 a = F_t - f_1 g m_1 \cos \alpha_1 - m_1g \sin \alpha_1 \]
\[ m_2 a = m_2 g \sin \alpha_2 - f_2m_2 g \cos \alpha_2 - F_t \]
\begin{figure}[hbtp]
\centering
\includegraphics[width=0.6\textwidth]{../User/sepal10.png}
\caption{Constraint 04}
\end{figure}
এই বীভৎসকে মানুষ করে, 
\[ m_1 a = m - n - c \]
\[ m_2 a = p - q - m \]
দুর্গম পথ পেরিয়ে, 
\[F_t = \frac{m_1 m_2 (\sin \alpha_1 + f_1 \cos \alpha_1 + \sin \alpha_2 - f_2 \cos\alpha_2)}{m_1 + m_2}  g \]
এর থেকে না হয় আমরা $f_3 \neq 0$ নিয়ে ভাবা শুরু করি!  \\
এই লম্বা পদ্ধতিতে সমাধান বের করাটা খুব জরুরি না, তবে এ থেকে কিন্তু আমরা ইতিমধ্যে ভালই ধারণা পেয়ে গেছি যে একটা General Problem এর থেকেই সব গুলো প্রবলেমের উদ্ভব হচ্ছে, আমরা দেখলাম যে ফাইনাল উত্তরেও কিন্তু আমাদের সবচাইতে সহজ constraint 01 এর $\frac{m_1 m_2}{m_1 + m_2}g$ বিরাজমান। এই ভাবে অনুশীলন করে বেশ খানিকটা পারদর্শী হয়ে ওঠা যায়। \\
%
কোনো এক প্রবলেমে যদি এমনটা থাকে যে একটা গোলকের হিসাব করতে হবে, বা গোলকের মধ্য দিয়ে সমাধানে পৌছাতে হবে, আমরা গোলকের যায়গায় একটা কিউব কিংবা অন্যকোনো সাধারণ আকৃতির বস্তু কল্পনা করে constraint বসিয়ে এগোতে পারি, অবশ্য পরে আমাদের এই constraint তুলে দিতে হবে, তবে সহজ করে ফেলা একটা প্রবলেম থেকে কঠিনের কিছু না কিছু আমরা পেয়েই যাই। \\
তবে একটা উপায়, বলতে গেলে একটা মেকানিক্সের আস্ত টপিক পড়ার পর হয়ত এ পর্যন্ত পড়া কোনো কিছু করে ছাড়াই আমরা প্রবলেম সল্ভ করে ফেলতে পারি!\footnote{এটা নিয়ে আস্ত বই লেখা যাবে, আমরা পরে এই যাদুকরি উপায় নিয়ে বিস্তৃত আলোচনা করবো, কারণ আমাদের গণিতের লিমিটের  সামনে এটা মোকাবিলা করা কষ্টকর ঠেকবে। \emph{Introduction to Classical Mechanics} by David Morin এটার খোজ পাওয়া যাবে, তবে Jargon Alert!}


 









\section{Revision Problems} 
এ পর্যন্ত আমরা যত আইডিয়া শিখেছি, মূলত সেসবের ব্যবহারের জন্য প্রবলেমগুলো দেয়া হলো। ইংরেজি পদার্থবিজ্ঞান বোঝার লক্ষে বাংলা এবং ইংরেজি দুটোই দেয়া হলো, এবং শেষে সব ইংরেজিই দেয়া হলো। অন্তত ৮০ শতাংশ সমস্যা করার চেষ্টা করবে। \\
\\
\textbf{Problem 01:} A vessel has a mixture of nitrogen ($m_1 = 7g$) and carbon dioxide ($m_2 = 11g$) at a temperature $T = 290K$ and pressure $p_0 = 1.05 \times 10^5$. Find the density of the mixture.\\
\textit{একটা কনটেইনারে নাইট্রোজেন এবং কার্বন ডাইওক্সাইডের এই এই ভরের মিক্সচার (বাংলা?) আছে, চাপ এত এত তাপমাত্রা এত এত, এবার এর ঘনত্ব বের করো।}\\
\\
\begin{figure}[hbtp]
\centering
\includegraphics[scale=0.4]{../User/sepal11.png}
\caption{Problem 00}
\end{figure}
\textbf{Problem 00:} Each of the masses are $m_1,m_2,m_3$ and they are in the system (configuration\footnote{যেভাবে সবকিছুকে অবস্থিত করা হয়েছে}) as shown in the figure, with the pulleys\footnote{কপিকল} frictionless. Find the acceleration of the $m_1$. What will happen if the $m_2 = \infty$ ? \\
\textit{(১)ছবি দেখে $m_1$ এর ত্বরণ বের কর। (২) $m_2 = \infty$ হয়ে গেলে কি হবে?}\\
\\
\textbf{Problem 00:} The temperature as a function of time of a material is shown by,
\[ T(t) = T_0 [ 1 + \alpha (t - t_0) ] ^{\frac{1}{4}} \]
Find the heat capacity of the material if power given is $P$. $\alpha, t_0, T_0$ are constant. \\
\textit{একটা বস্তুর সময়ের সাথে তাপমাত্রার ফাংশন দেয়া আছে। প্রতি সেকেন্ডে কত শক্তি দিতে হচ্ছে তা জানা আছে, এবার বস্তুর তাপমাত্রা এক ডিগ্রি বারাতে কত শক্তি লাগে, হিট ক্যাপাসিটি (বাংলা?) বের কর। $\alpha, t_0, T_0$ ধ্রুবক।}\\
\textbf{Hint:} There is a relationship between Energy (heat given) and the increment of Temperature. It's derivative gives the heat power that is known. This can be used simply without hasty assumptions.










 





 



























\end{document}
