\documentclass[10pt,a4paper]{article}

\usepackage{amsmath}
\usepackage{amsfonts}
\usepackage{amssymb}
\usepackage{graphicx}
\usepackage[left=2.5cm,right=2.5cm,top=2cm,bottom=2cm]{geometry}
\title{Beginning of Maths - Part 0 }
\author{Ahmed Saad Sabit}
\begin{document}
\maketitle
\section{Numbers - Adding and Subtracting}
\textbf{Problem:} Why do we need numbers?\\
\textbf{Solution:} To count. \\
\textbf{Problem:} How is counting done?\\
\textbf{Solution:} Counting is done by numbers as follows. 
\[0, 1, 2,3,4,5,6,7,8,9,10,11,12,13,14,15,....\] 
\textbf{Problem:} There are 2 balls from Adidas and 3 balls from other company. How many balls are there in total? \\
\textbf{Solution:} There are $2+3=5$ balls in total. \\
\textbf{Problem:} There is a folder in your computer that has 100 pictures. How many pictures will remain if you delete 10 pictures? \\
\textbf{Solution:} After deleting 10 pictures, the total number of pictures in the folder is reduced by 10. This means that, 
\[ 100 - 10\]
If we count, then,
\[ 100 - 10 = 90 \]
So there will be 90 pictures left.
\section{Numbers - Dividing and Multiplying}
\textbf{Problem:} You have 3 friends who have 10 Pokemon game card each. How many cards are there in total? \\
\textbf{Solution:} There are 3 friends with 10 cards. Total number of card is thus,
\[ 10+10+10 = 20 +10 = 30 \]
But this also means that, 
\[ 10+10+10 = 3 \times 10 = 30 \]
So there are 30 cards.\\
\textbf{Problem:} There is a number that can be found if you add 3 four times with itself. What is the number?\\
\textbf{Solution:} Let's add 3 four times with itself. 
\[3 + 3 + 3 + 3 =  6 + 3+3 = 9+3=12 \]
But this also means,
\[ 4 \times 3 = 12 \]
Last two problems is the idea of Multiplication. The number is 12.\\
\textbf{Problem:} There is a 20 meter long stick. What will be the length of the stick if we cut it in 10 equal pieces?\\
\textbf{Solution:} We have to directly divide this. 
\[ \frac{20}{10} = 2 \]
Thus, the length of each small part will be 2 meter. This makes sense. If you add ten 2 meter small parts, the total length will be 20. You can visualize this by,
\[ \frac{20}{10} \times 10 = 2 \times 10 \]
This is, 
\[ 20 = 2 \times 10 \]
Division is the reverse of multiplication. These come extremely frequently in Physics.
\section{Algebra - Math of Unknowns and Variables}
\textbf{Problem:} Suppose there is an unknown number of pens in a bag. If we take away $5$ pen from the bag, then there are $19$ pens remained in the bag. How many pens were there initially? \\
\textbf{Solution:} Let there be an unknown $\square$ pens. Taking away $5$ pen means, 
\[ \square - 5 \]
But after this, then, there remains 19 pens, this means,
\[ \square - 5 = 19 \]
Add +5 in both side of the equal sign. 
\[ \square - 5 + 5 = 19 + 5 \]
Then $-5 +5 = 0$. And, $19 + 5 = 24$. This gives,
\[ \square = 24 \]
This tells that there were 24 pens in the bag. \\
\textbf{Problem:} There is $200$ taka in a personal account. If we take $20,30,40,50$ taka in separately, how much will remain? \\
\textbf{Solution:} Let us take $\square$ amount of money, then remain, 
\[ 200 - \square \]
This square will take various values, 
\begin{align*}
& 200 - \square = \\
& 200 - 20 = 180 \\
& 200 - 30 = 170 \\
& 200 - 40 = 160 \\
& 200 - 50 = 150 
\end{align*}
\textbf{Problem:} There are 2 tanks of water. One tank is big and another tank is small. The small tank has some unknown volume of water.

The bigger one has water that is equal to the volume multiplied by the volume of the small tank. How much water is in the big tank? \\
\textbf{Solution:} Let there be unknown $\square $ amount of water in small tank. Then, in the bigger tank, there is,
\[ \text{Water volume in bigger tank} = \square \times \square \]
The $\square \times \square$ is written in the form $\square ^2 $

So,
\[ \text{Water volume in bigger tank} =\square ^2 \]
Mathematicians will soon use $x$ instead of a $\square$ box.
















\end{document}