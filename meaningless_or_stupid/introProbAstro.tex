\documentclass[11pt,a4paper]{article}
\usepackage{fontspec}
\setmainfont{Source Serif Pro Light}
\usepackage{amsmath}
\usepackage{amsfonts}
\usepackage{amssymb}
\usepackage{graphicx}
\usepackage{geometry}[top=2cm, right = 6.5cm, left = 6.5 cm, bottom = 0.5 cm]
\usepackage{wrapfig}
\author{Ahmed Saad Sabit}
\title{Physics Problems for Introduction to Astronomy}
\begin{document}
\maketitle
\begin{figure}[hbtp]
\centering
\includegraphics[width=0.4\textwidth]{../User/doiskAndPLANE.png}
\caption{The Whole Setup, the Point $P$ points towards the point $O$}
\end{figure}
\textbf{Problem:} There is a disk of radius $r_0$. It is on a plane surface. This disk has two motions, it spins about it's own center, and it also rotates about a point $O$. The radius of the circle, made by the center of the disk rotating around $O$ is $R$.

The period of rotation about the point $O$ is $T$ and the period of spin of the disk about it's own axis is $t_0$. On the disk, there is a point $P$, which is in the Edge. 
\begin{figure}[hbtp]
\centering
\includegraphics[width=0.4\textwidth]{../User/disk.png}
\caption{The disk, notice a random point $P$ at the Edge}
\end{figure}
This point $P$ initially points towards the point $O$, then it starts it's motion (neglect any friction and acceleration). \textsf{Find the time interval after which point $P$ shall again point towards point $O$.}

To do this, as a hint, see the distance traveled by the disk to make a full spin. Then, the line made by disk center and $P$ and the line made by disk center and $O$ shall make an angle. Use this idea.
\end{document}