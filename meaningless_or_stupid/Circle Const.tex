\documentclass[10pt,a4paper,landscape]{article}
\usepackage[utf8]{inputenc}
\usepackage{amsmath}
\usepackage{amsfonts}
\usepackage{amssymb}
\usepackage{graphicx}
\usepackage[left=2cm,right=2cm,top=2.8cm,bottom=2cm]{geometry}
\author{Ahmed Saad Sabit}
\title{Constructions on Circle Geometry}
\begin{document}
\maketitle
\section{Constructions}

\subsection{Drawing a tangent to a circle from a Point outside}

The methods are straightforward.
\begin{itemize}
\item Draw the center to the point - line. 
\item Draw the perpendicular bisector.
\item Take the center in the bisected point and draw a circle with radius upto the point
\item The circle cuts the main circle. Join the points and we have a tangent.
\end{itemize}

\begin{figure}[hbtp]
\centering
\includegraphics[width= 0.4\textwidth]{F:/User/const1.jpg}
\caption{Single tangent}
\end{figure}

\subsection{Constructing a Circumscribing Circle of a triangle}

Similiar procedure.\textbf{Circumcircle} is the circle that joins the vertex of the triangle.
\begin{itemize}
\item Draw the perpendicular bisector of any two side of the triangle.
\item The intersection of the bisector is the center of our sought circle, draw a segment joining a vertex. That is the radius.
\item What? Draw the circle and we are done!
\end{itemize}
\begin{figure}[hbtp]
\centering
\includegraphics[width =0.5\textwidth]{../User/const2.jpg}
\caption{Circumcircle}
\end{figure}
\newpage
\subsection{Construction of an Inscribed Circle}
\textbf{Inscribed Circle} is the one just fitted inside a triangle, also the circle with the largest area that can be taken by the same triangle.
\begin{itemize}
\item Draw the angle bisector of any two vertex of the triangle. 
\item The intersection is the center of the sought circle. Taking the ideal radius is the tricky part. I rather exploit the fact in GeoGebra.
\item The radius has to be \emph{Perpendicular} on any of the line of the triangle. Take an arc by an arbitrary radius and cut a line, I did the base one.
\item The two points cut on the line is then used to build a perpendicular bisector. \emph{Mark the bisection}.
\item Draw a line from incircle center to the marked bisection point. Use it as the radius and we have the circle.
\end{itemize}

\begin{figure}[hbtp]
\centering
\includegraphics[width=0.4\textwidth]{../User/const3.jpg}
\caption{Incircle}
\end{figure}

\section{Statistics}
\subsection{Simple Mean Average}
For the frequency and the value,
\[ \bar{x} = \frac{1}{n} \,	\sum_{i=1}^{k} f_i x_i \]

\subsection{Short Cut Mean}
Take these
\begin{itemize}
\item Class Interval
\item Class Mid Value
\item Frequency
\item Step deviation
\item Frequency $\times$ Step deviation
\end{itemize}
The step deviation is given by (mid val - approx mean)/(class int)
\[ u_i = \frac{x_i - a}{h}
\]
And the mean is found by,
\[ \bar{x} = \frac{ \sum f_i u_i      }{n} \times h
\]

\subsection{Median}
Median is the middle term or the average of the middle two terms. For a huge set of data,
\begin{equation}
\bar{M} =
L +
\left( \frac{n}{2}  -  F_{\text{cumulative of previous}} \right) \times \frac{h}{f_{\text{median class freq}}}
\end{equation}
Reconning this is tricky as it gets. Memorize that \textbf{Cumulative to Prev. , Freq to the Med}. \textbf{L} is the lower limit to the median class.

\subsection{Mode}
Mode is the data that is \emph{repeated the maximum times}.
\begin{align*}
\dot{M} = L + \frac{f_1}{f_1 + f_2} \times h \qquad				 	
		& f_1 = \text{frequency of mode class - frequency to the previous} \\
	    & f_2 = \text{frequency of mode class - frequency to the next} \\ 
\end{align*}

\subsection{Central Tendency}
If  the disorganized data of statistics as arranged according to the value, the tends to cluster round near any central value. Moreover, abundance of data id observed in a single class when these data are presented in some frequency distribution table. \textbf{The tendency of \emph{data} to cluster around the central value is known as the Central Tendency.}





\end{document}


























